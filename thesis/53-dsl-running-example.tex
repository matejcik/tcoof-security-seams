\section{Implementing the Running Example}
\label{dsl:example}

%%%

\subsection{Overview}

Our scenario consists of workers, which are assigned to projects; workrooms, which are
also assigned to projects; and lunchrooms, which are unassigned. Workers, workrooms, and
lunchrooms will be represented as components. Two distinct sub-problems exist, each with
its own parameters: assignment of workrooms and assignment of lunchrooms. These can be
naturally described as separate ensembles.

\medskip

When the building is open, workers are allowed to enter all workrooms assigned to their
project. We will create an ensemble for every project, and this ensemble will have the
following roles:
\begin{itemize}
    \renewcommand\labelitemi{--}
    \setlength\itemsep{0em}
    \item \textit{project workers}, inhabited by all workers for that project
    \item \textit{project rooms}, inhabited by all workrooms for that project
\end{itemize}
The ensemble will grant all project workers access to all project rooms.

\medskip

Lunchrooms open at lunch time. Workers can indicate that they are hungry, which we
represent as a knowledge field on the worker component. We would like to collect hungry
workers in small ensembles, each granting access to a single lunchroom. To accomplish
that, we will create an ensemble for every lunchroom, with the following roles:

\begin{itemize}
    \renewcommand\labelitemi{--}
    \setlength\itemsep{0em}
    \item \textit{occupants}, inhabited by all workers currently in the room, plus all
    workers that have previously been assigned to the room
    \item \textit{assignees}, inhabited by hungry workers who are not yet assigned
\end{itemize}
The ensemble will attempt to collect assignees from the pool of all hungry workers, up
until room capacity is filled. An additional constraint is that every member of this
ensemble must be assigned to the same project. That means that if there are existing
occupants, new assignees must have the same project as them. If the room has no current
occupants, new assignees can be selected from any project.

In addition, not all seatings are equally good. We want to use lunchrooms sparingly:
given the choice between putting a worker into an empty lunchroom and an occupied one,
the occupied should be picked, so that we keep the empty lunchroom available for other
projects.

Once a satisfactory solution has been found, the ensemble will allow both occupants and
assignees to enter the lunchroom, and notify new assignees that a seat was found for
them.

%%%

\subsection{Implementation}

Following this description is a full listing of the policy for the running example,
including definitions of components.

Lines~\ref{ex:components:start}--\ref{ex:components:end} define component types. All
rooms are of common supertype \cc{Room}. Apart from name, the \cc{LunchRoom} has a
knowledge field \cc{capacity}, stating its maximum occupancy.

\cc{Worker}s have three knowledge fields: their assigned \cc{project} (specified in
constructor), their \cc{hungry} status, and their current \cc{location}. A helper method
\cc{isInLunchRoom} returns \cc{true} if the worker's location is a lunchroom, as an
example of ``computed knowledge'' from section~\ref{dsl:c:components}

Next, case classes are defined on lines~\ref{ex:casecls:start}--\ref{ex:casecls:end}.
Projects do not need to be components, but we still need structured information about
them, particularly their list of assigned workrooms. We also create a case class for
lunchroom assignment notifications.

Scenario class definition starts at line~\ref{ex:scenario}. Its inputs are lists of
projects, workers, workrooms, and lunchrooms. An instance of this scenario class could
represent a workday in a single building. Opening and closing times are part of the
security policy and are hard-coded at lines~\ref{ex:times:start}--\ref{ex:times:end}.

Line~\ref{ex:now} defines a variable to represent current time. In a real-life
deployment, this might be represented by a reference to system clock; however, for our
testing, it is useful to set the current time explicitly.

We also define a helper attribute \cc{workersByProject} for easier access to lists of
workers from the same project.

Class \cc{RoomAssignment} at line~\ref{ex:root} represents the policy root. It defines
one pseudo-role, \cc{hungryWorkers}, which is inhabited by all workers who are (1)
hungry, (2) not currently in a lunchroom, and (3) not already assigned to a lunchroom.
This will be the pool from which the lunchroom ensembles select candidates. Furthermore,
the root ensemble contains definitions of the two sub-ensembles for our two
sub-problems.

Workroom ensemble is described by the class \cc{WorkroomAssignment} on
line~\ref{ex:workroom}. It takes a project definition as a parameter. Its situation
predicate specifies that it is only active between \cc{BuildingOpenTime} and
\cc{BuildingCloseTime}. The statically-assigned role \cc{projectWorkers} is simply the
list of workers for the ensemble's project. The grant at line~\ref{ex:workroomgrant}
allows all \cc{projectWorkers} to enter all \cc{workrooms} of the project.

\medskip

Lunchroom ensemble is represented by the \cc{LunchroomAssignment} class at
line~\ref{ex:lunchroom}, and takes a lunchroom as an argument. Like the workroom
ensemble, it also has a situation predicate; this time specifying that it applies
between \cc{LunchOpenTime} and \cc{LunchCloseTime}.

The first role, \cc{occupants} at line~\ref{ex:occupants}, is a statically determined
list of workers that are either (a) already assigned to the room, as determined by the
appropriate \cc{LunchRoomAssigned} notification, or (b) physically present in the room,
as indicated by their \cc{location} attribute.

We use \cc{occupants} to calculate \cc{freeSpaces}, the number of remaining seats in the
room. Then, at line~\ref{ex:assignees}, we define the \cc{assignees} role as a
dynamically selected subset of \cc{hungryWorkers}, limiting its size to the number of
free seats.

The \cc{eaters} role at line~\ref{ex:eaters} is a union of \cc{occupants} and
\cc{assignees}. It is not a meaningful role in the ensemble, but implementation-wise, it
is needed to specify the constraint on the next line: all \cc{eaters} must have the same
value of their \cc{project} attribute.

Lines~\ref{ex:utility:start}--\ref{ex:utility:end} define the utility expression. Fuller
rooms are preferred, which is expressed by setting the utility to the square of total
number of used seats. This way, a solution that places two workers in the same lunchroom
is measured as better than a solution that places each of them in a separate room.

Finally, \cc{assignees} are notified of their assignment at line~\ref{ex:notify},
and line~\ref{ex:lunchroomgrant} allows all \cc{eaters} to enter the room.

\medskip

Back at the \cc{RoomAssignment} level, lines~\ref{ex:rules:start}--\ref{ex:rules:end}
create instances of the sub-ensembles. An instance of \cc{WorkroomAssignment} is
generated for every project, and an instance \cc{LunchroomAssignment} is generated for
every lunchroom. The \cc{rules} call configures the sub-ensembles to be selected
whenever their situation predicate is true, i.e., they are always active in their
specified time-frames.

The constraint at line~\ref{ex:globalconstraint} ensures that all instances of the
\cc{assignees} role are disjoint, or in other words, that no worker can get a seat
reservation in more than one lunchroom at the same time.

Line~\ref{ex:policy} instantiates the policy object and makes it available as an
attribute.

%%%
\pagebreak

\subsection{Source Code}

\begin{lstlisting}[style=ensembles]
// Different types of rooms |\label{ex:components:start}|
abstract class Room(name: String) extends Component {
  name(s"Room:$name")
}
class LunchRoom(name: String, val capacity: Int)
  extends Room("Lunch" + name)
class WorkRoom(name: String)
  extends Room("Work" + name)

// Worker assigned to a project, can be hungry or not
class Worker(id: Int, val project: Project) extends Component {
  name(s"Worker:$id:${project.name}")
  var hungry = false
  var location: Option[Room] = None

  def isInLunchRoom: Boolean =
    location.map(_.isInstanceOf[LunchRoom]).getOrElse(false)
} |\label{ex:components:end}|

// Project with pre-assigned workrooms |\label{ex:casecls:start}|
case class Project(name: String, workrooms: Seq[WorkRoom])

// Notification for lunchroom assignment
case class LunchRoomAssigned(room: LunchRoom) extends Notification |\label{ex:casecls:end}|

class LunchScenario(val projects: Seq[Project],  |\label{ex:scenario}|
                    val workers: Seq[Worker],
                    val workrooms: Seq[WorkRoom],
                    val lunchrooms: Seq[LunchRoom]) {
  // Opening times of the building and of the lunchrooms
  val BuildingOpenTime  = LocalTime.of( 7, 30)  |\label{ex:times:start}|
  val BuildingCloseTime = LocalTime.of(21,  0)
  val LunchOpenTime     = LocalTime.of(11, 30)
  val LunchCloseTime    = LocalTime.of(15,  0)  |\label{ex:times:end}|

  val DefaultNow = LocalTime.of(8, 42)
  var now = DefaultNow  |\label{ex:now}|
  
  // mapping projects to lists of workers
  val workersByProject = workers.groupBy(_.project)

  class RoomAssignment extends Ensemble { |\label{ex:root}|
    name("assign workers to projects and rooms")

    // list of all hungry workers waiting for a lunchroom
    val hungryWorkers = workers.filter { w =>
      w.hungry &&
      !w.isInLunchRoom &&
      !w.notified[LunchRoomAssigned]
    }

    // Each worker assigned to a project can access all workrooms
    // assigned to that project when the building is open.
    class WorkroomAssignment(project: Project) extends Ensemble { |\label{ex:workroom}|
      name(s"assign workrooms to workers on project ${project.name}")

      situation { (now isAfter BuildingOpenTime) &&
                  (now isBefore BuildingCloseTime) }

      val projectWorkers = workersByProject.getOrElse(project, Seq.empty)
      allow(projectWorkers, "enter", project.workrooms) |\label{ex:workroomgrant}|
    }

    // Each hungry worker will get an assigned lunchroom so that
    // no lunchroom is over capacity and workers from different
    // projects do not meet in the same lunchroom.
    class LunchroomAssignment(room: LunchRoom) extends Ensemble { |\label{ex:lunchroom}|
      name(s"assign workers to lunchroom ${room.name}")

      // Only activate when lunchrooms are open
      situation { (now isAfter LunchOpenTime) &&
                  (now isBefore LunchCloseTime) }

      // list of previously assigned workers
      val occupants = workers.filter { w =>  |\label{ex:occupants}|
        w.notified(LunchRoomAssigned(room)) |\textbar\textbar|
        w.location.contains(room)
      }

      // newly-assigned hungry workers must fit into free space
      val freeSpaces = room.capacity - occupants.size
      val assignees = subsetOf(hungryWorkers, _ <= freeSpaces) |\label{ex:assignees}|

      val eaters = unionOf(occupants, assignees) |\label{ex:eaters}|
      constraint { eaters.allEqual(_.project) }

      // Set the solution utility to square of the number of occupants,
      // i.e., prefer many workers in one room over few workers in many rooms
      utility { |\label{ex:utility:start}|
        val occupied = assignees.cardinality + occupants.size
        occupied * occupied
      } |\label{ex:utility:end}|

      // grant access rights and notify newly selected hungry workers
      notify(assignees, LunchRoomAssigned(room)) |\label{ex:notify}|
      allow(eaters, "enter", room) |\label{ex:lunchroomgrant}|
    }

    val workroomAssignments =  |\label{ex:rules:start}|
      rules(projects.map(new WorkroomAssignment(_)))
    val lunchroomAssignments =
      rules(lunchrooms.map(new LunchroomAssignment(_))) |\label{ex:rules:end}|

    // ensure that a worker is not assigned to more than one lunchroom
    constraint(lunchroomAssignments.map(_.assignees).allDisjoint) |\label{ex:globalconstraint}|
  }

  val policy = Policy.root(new RoomAssignment) |\label{ex:policy}|
}
\end{lstlisting}
\pagebreak
