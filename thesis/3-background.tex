\chapter{Background and Related Work}
\label{background}

\section{Ensemble-Based Architectures}
\label{background:ensembles}

The complexity of managing a system grows with its size. This has traditionally been
handled through hierarchical decomposition, component-oriented programming, and similar
techniques. With the rise of the Internet of Things, however, we are reaching the limits
of these approaches. If the environment consists of many independent agents with no
clear functional hierarchy, it is impractical to impose a top-down view of control.
There is no single point of access to a ``swarm'' of sensors, smart things, and
cyber-physical systems in general, especially when these are distributed over a large
area with no promise of reliable connectivity and continued availability. The use-cases
are new, too: clients want to access the system from different locations and
perspectives, use different services in different configurations, and take advantage of
the inherent dynamicity.

The problem is partially solved through \textit{Service-Oriented Architectures}
\citep{SOA}. SOA revolves around modular, dynamically discovered and dynamically bound
services. Unlike a typical ``hard-coded'' component-based system, SOA can deal with
small services that are randomly appearing and disappearing from the network. There are
limitations, though. SOA is still a top-down approach that assumes its parts are
discrete services. This is not necessarily true with IoT. Consider a home sensor
network. SOA can expose every sensor as its own service, but then it falls to the client
to locate, e.g., all sensors in a single room. Alternately, the whole network could be a
single service, but then it needs to have ``zoom in on a single room'' as a feature.
This gets complicated when the query changes, e.g., when locating all temperature
sensors. The issue is more pronounced when considering a heterogeneous city-wide network
of smart things. Also, service discovery starts being a difficult problem when the
number of services grows to tens of thousands.

A whole other issue is agent collaboration. Certain problems lend themselves to
distributed approaches --- e.g., self-driving cars collaborating to find parking spaces
along their driving routes. While it is possible to create a centralized service for
this task, it would be more practical to have the agents collaborate directly and
locally. This is not a problem that SOA can solve, and indeed it is unsuitable for any
kind of top-down approach. Instead, we need to look towards multi-agent systems to
manage the collaboration, and possibly some other approaches that will allow the agents
to locate each other, establish communication links, organize and reorganize.

To restate: we are experiencing an explosion of the size of cyber-physical systems and
the number of independent computing devices, and traditional approaches fail to use
these systems efficiently. There is a growing need to organize highly dynamic systems
that consist of many independent agents. Not only is a different approach required, we
actually need different abstractions.

\medskip

One such abstraction is the \textit{ensemble}. An ensemble is a loose coalition of
\textit{components} formed around a shared goal. A component can be any kind of moving
part: in a software system, a component would be a module, object, service, any piece of
code performing a particular functionality. In the world of cyber-physical systems, a
component can be a computer, an IoT device, a drone, etc. As we show in this work, we
can even model humans, rooms and other non-computational entities as components.

Members of an ensemble inhabit specific \textit{roles}, which are usually described in
terms of capabilities or services a component can provide. This decouples concrete
components from the ensemble. For example, an ensemble providing environmental data for
a HVAC\footnote{Heating, Ventilation, Air Conditioning} system could have a
``thermometer'' role, a ``CO\textsubscript{2} detector'' role, and an ``uplink'' role
which handles Internet connectivity. These would be inhabited by appropriate devices,
without regard to which specific device is providing which service.

A single component can perform multiple roles in the same ensemble (e.g., an
Internet-capable thermometer can function both as a sensor and as an uplink), or be a
member of several different ensembles (the same uplink can serve multiple sensor suites
at the same time). It is also conceptually simple to replace a component with a
different one in the same role, in case the former member becomes unavailable or no
longer fits the ensemble's parameters.

Ensembles are composable; an ensemble can contain any number of sub-ensembles, which in
turn can be composed of more sub-ensembles. Sometimes it is useful to set up a role
which can be inhabited by a sub-ensemble.

\medskip

One of the key features of an ensemble-based architecture is its high level of
dynamicity. Ensembles are formed and dissolved when needed. Similarly, membership in an
ensemble is driven by current context and conditions, not by any sort of explicit
registration. Therefore, the natural way to specify ensembles is declarative. We can use
predicates over component properties and the local conditions. E.g., a smart parking lot
can define an ensemble as ``parking meter, plus all vehicles seeking parking within 10
minutes from it.''

This meshes well with ideas of autonomic computing~\citep{IBM2003}: individual components
could be able to determine ensemble membership based on their own knowledge, without a
central coordinator. Alternately, when a global perspective is available, a coordinator
can use all available information to optimize ensemble membership and role assignment,
and can be flexible in responding to change.

Due to its ephemeral nature, it is difficult to work with an ensemble ``from the
outside''. In an autonomic system, ensembles might not even be visible from the outside,
their existence known only to the individual members. And even with a central
coordinator, there are some challenges: for example, an ensemble can dissolve at any
time and its constituent components can be reassigned. There is no way to maintain a
persistent reference to a particular ensemble or a role within it. Instead, the ensemble
system should be considered self-organizing and self-governing. A user of the ensemble
system can post goals, and the system will ensure that ensembles are formed to resolve
that goal.

\subsection{Related Work}

Ensembles present a perspective which differs significantly from traditional views of
component systems. For this reason, new methodologies and paradigms are being designed
for efficient programming of ensemble systems.

\medskip

\textit{Software Component Ensemble Language} was introduced in \citep{SCEL2013} in 2013
and refined a year later in \citep{SCEL2014}. SCEL is designed as a ``kernel'' language,
a mostly abstract grammar that is supposed to be a building block for more concrete and
full-featured languages. Its paradigm is built on four concepts: \textit{knowledge} of
individual components; \textit{behaviors} operating over knowledge bases;
\textit{aggregations} as collections of components; and finally \textit{policies} that
can control and adapt execution of behaviors. Each component is represented as an
interface exposing its knowledge base, available behaviors, and governing policies.

Despite having ``ensemble'' in the name, there is no explicit notion of an ensemble nor
a role. They exist implicitly, via the ability to control targets of actions with
logical predicates.

Interestingly, the SCEL paper~\citep{SCEL2014} shows an access control sub-language as
an example of the policy concept. However, the language is used to protect processes on
the level of one component, and its capability to consider broader situation context is
limited, so it is of little interest to the topic of this work.

\medskip

HELENA, from \textit{Handling massively distributed systems with ELaborate ENsemble
Architectures} \citep{HELENA2014} \citep{HELENA2016}, is a rigorous formal approach to
ensemble-based systems, focused primarily on the role concept. The goal of the design is
to enable formal verification of ensemble behavior. Role operations are specified by
a~labeled transition system, and ensembles are modeled as automata. Components are
considered resources for the roles they can fulfill, but are themselves passive; any
active behavior is initiated through the role abstraction.

As the HELENA approach examines the problem from the role level, the process of forming
an ensemble is left unexplored. Finding components to inhabit roles is assumed to be
implicit.

\medskip

The \textit{Distributed Emergent Ensembles of Components} (DEECo) model
\citep{DEECo2016} is a more practice-oriented approach, tailored for use in IoT
applications. It defines a paradigm of \textit{ensemble-based component systems}, where
components, roles, and ensembles are all first-class objects. The DEECo paper also
discusses the question of reliability of communication and presumes use of mesh
networking.

Membership in an ensemble is determined by a \textit{membership condition} over
components' knowledge attributes. Component roles are modeled as interfaces, prescribing
the knowledge attributes that must be available to the ensemble. Ensemble definition
further specifies what knowledge is exchanged between members, when, and how often.

Ensemble formation is initiated by a \textit{coordinator}. Each ensemble specifies a
single coordinator role. The component which inhabits that role will look for other
members via local or routed broadcast on the network, and facilitate the knowledge
exchange processes. While the paper doesn't elaborate on this, the concept is
nevertheless important, as it basically imposes locality on the search for members.

\medskip

A follow-up paper~\citep{dynamic2017} expands on the idea of search locality, adds
discussion of performance issues, and introduces the concept of filtering as a way to
limit the number of components in consideration. It also adds a concept of
\textit{fitness}, a quality score of the formed ensemble. For example, it is possible to
specify that an ensemble consisting of nearby components is a better candidate than an
ensemble whose components are far from each other.

In~\citep{tcoof2017} an architecture definition language called \textit{Trait-based
Coalition Formation ADL} (TCOF-ADL) is presented. Its express goal is to facilitate
selection of components and formation of ensembles, while ensuring that formation
responsibilities are properly distributed between agents in the ensemble system. It
provides some basic predicates and allows components to be extended with ensemble
formation related traits such as location awareness, data prediction, and a statistical
evaluator. The language is implemented as a Scala-internal DSL and employs a constraint
solver to help with ensemble membership resolution.

A variant of TCOF-ADL, named TCOOF-Trust\footnote{The additional ``O'' in the name is
not explained. Presumably it comes from changing ``coalition'' to ``coordination'' in
the acronym.}, is presented in~\citep{isola2018}. The paper specifically deals with
application of ensemble concepts to access control, and this work is a direct follow-up
to the same research. The TCOOF-Trust language reuses some of the original constructs,
as well as the ensemble formation engine, and adds commands related to access control
decisions.

\section{Access Control in Dynamic Systems}
\label{background:access}

Historically, access control tended to be rule-based and local to specific resources,
such as files or concrete objects. In UNIX, for example, each file has an owner and a
group, and a fixed bitmap of permissions that can be granted or denied to the owner,
members of the group, or everyone else. An improvement to this system is an
\textit{Access Control List} (ACL) attached to an object, which can list an arbitrary
number of users, groups, and their permissions.

As system complexity grows, this is no longer a sufficient solution. Many contemporary
systems use some variant of \textit{Role-Based Access Control}~\citep{rbac1995} (RBAC).
Access control rules are expressed as tuples: (\textit{role}, \textit{operation},
\textit{subject}), authorizing members of \textit{role} to perform \textit{operation} on
a \textit{subject} resource. In addition, roles can be arranged in a hierarchy, so that
a role higher in the hierarchy inherits all permissions of lower roles. This naturally
fits human organizational hierarchies: e.g., a supervisor role will usually inherit all
permissions of the worker role, without need to specify them again.

A single user can be assigned to any number of roles. This provides an important
abstraction and simplifies user management, as it is now possible to grant or revoke
user privileges simply by modifying their role assignment. A more recent
\textit{Organization-based Access Control}~\citep{orbac2003} (OrBAC) model adds another
layer of abstraction: each of \textit{role}, \textit{operation} and \textit{subject}s in
the master policy can now be ``implemented'' by a corresponding object in a local
environment. This enables sharing policies between organizations at design-time and
collaboration at run-time, as two organizations can cross-authorize their users in
compatible roles.

As described, RBAC-like policies are not aware of wider context. Role $R$ has permission
to perform operation $O$ on subject $S$ at all times, regardless of situation or current
conditions. That is not good enough for many real-world scenarios, where context plays
an important role. For instance, a ``worker'' should have permission to ``enter'' the
``workspace'' --- but only during work hours.

Several approaches arose to close this gap. \textit{Context-Sensitive
RBAC}~\citep{contextkumar2002} adds context information to roles, operations, and
subjects, and attaches predicates over the contexts to individual access control rules.
Similarly, \textit{Context-Aware RBAC}~\citep{contextkulkarni2008} introduces
preconditions that can query context information before allowing the rule to apply. In
addition, a ``context guard'' predicate can be specified, which must hold true in order
for active sessions to remain open. In~\citep{contextrbac2004}, context constraints are
separated into those that can be evaluated at design-time and those that must be
evaluated at run-time, and a development methodology for constrained rules is presented.

In recent years, \textit{Attribute-based Access Control}~\citep{abac2014} (ABAC) is
gaining popularity because of its even greater flexibility. In ABAC, permissions are not
attached to roles, but to arbitrary expressions over attributes of users and subjects.
As an example, RBAC could be considered a special case of ABAC, with the only attribute
under consideration being the list of user's roles. But every object can have any number
of attributes and these can dynamically reflect current situation. Attributes of
environment, such as current date and time, can also be accessed.

A popular realization of the ABAC model is OASIS XACML~\citep{xacml2013} standard, which
is an XML-based policy language, and a more developer-friendly language
ALFA~\citep{alfa} that translates directly to XACML. Developers can describe rules using
a large number of built-in predicates, or they can define custom functions for more
complex situations.

\medskip

The main limitation of these approaches, which we will collectively call ``rule-based'',
is the actor-action-subject perspective. ABAC is capable of querying the situation, but
the query still originates from a static description of applicable actors, actions, and
subjects. In other words, rules can query context, but context cannot influence rules.

This makes rule-based approaches highly impractical in scenarios where the context does
in fact cause the rules to change. The implementation of such context-dependent security
policy is scattered across many different rules, which makes it hard to audit, maintain
and debug. In fact, it is difficult to enumerate the rules that should be affected in
the first place.

One example is dynamic global state which cross-cuts ordinary rules. For instance, a
``night policy'' might be different from a ``day policy''. Affected rules would need to
have two variants, and updating one of the policies couldn't be done in one place. In
addition, if global states can overlap (consider a ``lock-down state'' when every door
only opens for members of a designated responder group and nobody else), complexity can
grow exponentially. This growth manifests in every affected rule, possibly to the point
where it is more practical to go outside the ABAC system and simply maintain separate
manually deployed policy configurations.

In highly dynamic systems, such as IoT networks, the same limitation can be seen from a
different angle: it is burdensome to express every possible combination of actors,
actions, and subjects beforehand. Dynamic relationships arise, form, and dissolve
throughout the life of the system. In our running example, a person is not allowed to
enter a room if another person working on a different project is already inside that
room. Such rule can be expressed as a context query with some creative use of
attributes, but it is more natural to query the context first and dynamically form the
rule afterwards.

\medskip

A number of so-called ``adaptive security'' solutions and frameworks have been proposed
to get around this limitation, each with their own pros and cons.

dynSMAUG~\citep{dynsmaug2017} paper works with the concept of a \textit{situation}
distinct from \textit{context}. While context is a collection of information on the
actors and the environment at a specific point in time, a situation is a time frame ---
an interval in which some context predicates hold true, possibly including particular
sequences of events. Situations are defined separately from the security policy in a
SQL-like language Esper~\citep{esper}, and currently applicable situations are exposed
as attributes queryable by XACML/ALFA languages.

While the model is sound, the need to describe the policy in two phases and in two
different languages is impractical in terms of maintainability. Furthermore, the Esper
language is powerful but also very complex even for simple cases.

A self-learning model was presented in \citep{adaptivesec-bashar2012}. Protected assets,
operational goals and possible threats are modeled and then compiled to a fuzzy decision
network. This network can evaluate the state of the system at run-time and configure or
reconfigure applicable security policies. This solution, however, is limited to
reasoning about a closed system, with no consideration for independent agents.

The solution presented in \citep{saleemi_framework_2011} uses inference rules based on
logic programming to build higher-level context information from lower-level knowledge
(such as sensor readings). Resulting high-level knowledge is then used as a basis for
security actions. This enables the policy designer to make security decisions at the
appropriate level of abstraction.
