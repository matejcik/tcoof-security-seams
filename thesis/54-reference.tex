\section{Reference}
\label{dsl:reference}

Scala is a strongly typed language, so a Scala-internal DSL is also strongly typed. In
order to make this section more concise, however, we will be using simplified type
signatures. The following simplifications are used:

\begin{itemize}
\item Whenever a function has bounded type parameters, we omit them in favor of the most
general type.
\item If a function does not return a value, the return type \cc{Unit} is omitted.
\item Whenever a function has a variadic argument (denoted with an asterisk), at least
one argument must be provided.
\item For every function with a variadic argument of type \cc{T}, denoted as \dop{T*},
an overload exists that takes a single \cc{Iterable[T]} argument instead.
\item We use type name \cc{Role} for brevity, but that type does not exist. The actual
type is \cc{MemberGroup[Component]}.
\end{itemize}

Unabridged function signatures are available in the Scaladoc API documentation in the
\cc{apidoc} directory of the accompanying archive.

\medskip

For readers not deeply familiar with Scala, we point out the \textit{by-name parameters}
feature. Whenever an argument type is prefixed with \dop{=>}, it is not evaluated
immediately, but only when it is used. This allows us to write expressions that would
fail at ensemble definition time, but work fine when the framework executes them.

%%%

\newenvironment{dslsig}%
    {%
        \par\vspace{0.6em}\bfseries\ttfamily\raggedright
    }%
    {%
        \vspace{-0.2em}
    }%

\newenvironment{dsldesc}%
    {%
        \nopagebreak
        \setlength{\parindent}{0em}
        \setlength{\parskip}{0.3em}
        \begin{adjustwidth}{1cm}{}
    }%
    {%
        \end{adjustwidth}
    }%

%%%

\subsection{\texttt{Component} class}
\label{dsl:r:component}

\cc{Component} must be used as a superclass of every component type. Only one function
is available at declaration time:

\begin{dslsig}
name(nm: String)
\end{dslsig}
\begin{dsldesc}
    Set a name of the component. This is useful for debugging purposes, when printing
    out ensemble memberships.
\end{dsldesc}

\medskip

\noindent
Component instances also have several methods from trait \cc{Notifiable} for querying
received notifications:

\begin{dslsig}
notifications: Iterable[Notification]
\end{dslsig}
\begin{dsldesc}
    Return a collection of all \cc{Notification} instances received by this component.
\end{dsldesc}

\begin{dslsig}
notified(notification: Notification): Boolean
\end{dslsig}
\begin{dsldesc}
    Query a specific notification. Return \cc{true} if the exact specified notification
    was received by this component.
\end{dsldesc}

\begin{dslsig}
notified[N <: Notification]: Boolean
\end{dslsig}
\begin{dsldesc}
    Query a notification class. Return \cc{true} if any notification of type \cc{N} was
    received by this component.
\end{dsldesc}

%%%

\subsection{\texttt{Integer} type}

\cc{Integer} is a generic reference to an integer number whose value might not be known
until a solution is found. It is usually a result of operations on member group objects.
Basic arithmetic operators on \cc{Integer}s are overloaded to return \cc{Integer}s and
basic comparison operators are overloaded to return \cc{Logical}s. Implicit conversion
from \cc{Int} is available, so that it is possible to mix \cc{Integer} calculations with
standard Scala math.

One notable imperfection is that the equality operator \dop{==} cannot be overloaded in
Scala. For comparing values of \cc{Integer}s, use the triple-equals \dop{===} operator
instead. The operator \dop{==} will compare object identities and return a~boolean.

%%%

\subsection{\texttt{Logical} type}

\cc{Logical} is a generic reference to a truth value which might not be known until a
solution is found. It is usually the type of constraint operations. Basic boolean
operators on \cc{Logical}s are overloaded to return \cc{Logical}s. Implicit conversion
from \cc{Boolean} is available, so that it is possible to mix \cc{Logical} expressions
with statically evaluated booleans.

%%%

\subsection{\texttt{Ensemble} class}

The security policy consists of a nested series of classes deriving from \cc{Ensemble}.
Most of the ensemble definition happens in the body of \cc{Ensemble}, so this class
provides most of the available functions.

\begin{dslsig}
name(nm: String)
\end{dslsig}
\begin{dsldesc}
    Set a descriptive name of the ensemble. This is useful for code documentation and 
    for debugging purposes, when printing out ensemble memberships.
\end{dsldesc}

\begin{dslsig}
utility(util: => Integer)
\end{dslsig}
\begin{dsldesc}
    Assign an utility function to the ensemble. \cc{util} is an \cc{Integer} expression
    that is evaluated for each solution being tested. Refer to
    subsection~\ref{dsl:c:utility} for detailed semantics.
\end{dsldesc}

\begin{dslsig}
rules(ensembles: Ensemble*): EnsembleGroup
\end{dslsig}
\begin{dsldesc}
    Register sub-ensemble(s) with static activation. Sub-ensembles registered via this
    function \textit{must} be activated if possible.

    Each sub-ensemble must be registered with \cc{rules} or \cc{ensembles} to take part
    in the computation.
\end{dsldesc}

\begin{dslsig}
ensembles(ensembles: Ensemble*): EnsembleGroup
\end{dslsig}
\begin{dsldesc}
    Register sub-ensemble(s) with dynamic activation. Sub-ensembles that are registered
    via this function \textit{can} be activated by the solver, if that leads to a good
    solution.

    Each sub-ensemble must be registered with \cc{rules} or \cc{ensembles} to take
    part in the computation.
\end{dsldesc}

\pagebreak
\begin{dslsig}
oneOf(items: Component*): Role \\
oneOf(role: Role): Role
\end{dslsig}
\begin{dsldesc}
    Define a role inhabited by \textit{exactly one} of the specified components or
    inhabitants of the specified role.
\end{dsldesc}

\begin{dslsig}
allOf(items: Component*): Role
\end{dslsig}
\begin{dsldesc}
    Define a role inhabited by \textit{all} of the specified components.

    This function is useful for explicit conversion of components to role objects.
    However, in most cases, components and collections of components are implicitly
    converted to roles as needed. E.g., the following two ensemble definitions are
    equivalent:
\begin{lstlisting}[style=ensembles]
val members = for (_ <- 1 to 5) yield new Member

object WithAllOf extends Ensemble {
    val role = allOf(members)
    allow(role, "open", door)
}

object WithoutAllOf extends Ensemble {
    allow(members, "open", door)
}
\end{lstlisting}
\end{dsldesc}

\begin{dslsig}
subsetOf(items: Component*): Role \\
subsetOf(role: Role): Role \\
subsetOf(role: Role, cardinality: Integer => Logical): Role
\end{dslsig}
\begin{dsldesc}
    Define a role inhabited by a \textit{subset} of the specified components or
    inhabitants of the specified role.
    
    The optional argument \cc{cardinality} specifies a constraint on the subset's
    cardinality. It is a function that takes an \cc{Integer} argument, representing the
    subset's cardinality, and returns a \cc{Logical} result configuring whether the
    cardinality is valid. It is possible to use Scala's placeholder underscore as a
    shortcut, i.e.:

\begin{lstlisting}[style=snippet]
val role = subsetOf(components, _ < 10)
\end{lstlisting}
\end{dsldesc}

\begin{dslsig}
unionOf(roles: Role*): Role
\end{dslsig}
\begin{dsldesc}
    Defines a role whose members are a \textit{union of specified roles}. Specifically,
    any component that inhabits one of the \cc{roles} also inhabits the union role, and
    if a component inhabits the union role, then there exists at least one role in
    \cc{roles} which the component also inhabits. This is mainly useful for specifying
    constraints over collections of roles that would otherwise be difficult to express
    individually; e.g., total size of the union must not exceed a specified number.
\end{dsldesc}

\begin{dslsig}
allow(actors: Role, action: String, subjects: Role)
\end{dslsig}
\begin{dsldesc}
    Grant permission to each inhabitant of \cc{actors} role to perform \cc{action} on
    each inhabitant of the \cc{subjects} role.

    Through implicit conversions, it is possible to use a component or an iterable of
    components in place of any of the \cc{Role} arguments.
\end{dsldesc}

\pagebreak
\begin{dslsig}
deny(actors: Role, action: String, subjects: Role)
\end{dslsig}
\begin{dsldesc}
    Deny permission to each inhabitant of \cc{actors} role to perform \cc{action} on
    each inhabitant of the \cc{subjects} role.

    Through implicit conversions, it is possible to use a component or an iterable of
    components in place of any of the \cc{Role} arguments.
\end{dsldesc}

\begin{dslsig}
notify(targets: Role, message: Notification)
\end{dslsig}
\begin{dsldesc}
    Send a \cc{message} to each of \cc{targets}. The message is persisted
    across solver runs, and its presence can be queried when forming ensembles. See
    subsection~\ref{dsl:c:notifications} for detailed semantics and
    subsection~\ref{dsl:r:component} for query methods.

    Through implicit conversions, it is possible to use a component or an iterable of
    components in place of the \cc{Role} argument.
\end{dsldesc}

\begin{dslsig}
constraint(clause: => Logical)
\end{dslsig}
\begin{dsldesc}
    Set up a constraint that must be satisfied in every solution. The clause must be of
    type \cc{Logical}, because it is propagated to the constraint programming
    engine, where it limits the search space. Therefore, it is possible to use role
    object expressions as constraints.

    Multiple constraints can be specified in an ensemble and each one must be satisfied
    in a valid solution.
\end{dsldesc}

\begin{dslsig}
situation(predicate: => Boolean)
\end{dslsig}
\begin{dsldesc}
    Set up a situation predicate. The predicate is evaluated \textit{before} the solver
    starts processing the ensemble. If it evaluates to \cc{false}, the ensemble
    is excluded from the solution.

    \cc{situation} has no effect in the root ensemble.
\end{dsldesc}

%%%

\subsection{Member Groups}
\label{dsl:r:membergroups}

The base class \cc{MemberGroup[T]} maintains a collection of components or ensembles. A
so-called ``role object'' is in fact a \cc{MemberGroup[C <: Component]}. Ensemble groups
are represented by a subclass \cc{EnsembleGroup}, which performs additional handling
related to ensemble hierarchies. However, all functionality relevant to the DSL is
defined in the base class, and thus identical for roles and ensemble groups.

For simplicity, we will use the type name \dop{Member} to stand in for the member type
of a \cc{MemberGroup}.

The member group provides a notion of \textit{selected members} --- a subset of the
member collection which is considered part of the solution. For instance, a role created
with the \cc{oneOf} function will have exactly one selected member.

All \cc{Integer} methods can be used to build constraints with arithmetic and comparison
operators. All \cc{Logical} methods can be used as constraints directly, or combined
with other constraints using boolean operators. For simplicity, the descriptions use
terms like ``true if'' or ``false if'', but keep in mind that the truth value of
\cc{Logical} is only relative to a candidate solution.

\begin{dslsig}
selectedMembers: Iterable[Member]
\end{dslsig}
\begin{dsldesc}
    List of \cc{Member} instances that are selected for the solution.

    Throws an exception if no solution has been generated.
\end{dsldesc}

\begin{dslsig}
cardinality: Integer
\end{dslsig}
\begin{dsldesc}
    Cardinality of the group, i.e., the number of selected members.
\end{dsldesc}

\begin{dslsig}
contains(member: Any): Logical
\end{dslsig}
\begin{dsldesc}
    True if the specified member is selected.
\end{dsldesc}

\begin{dslsig}
containsOtherThan(member: Any): Logical
\end{dslsig}
\begin{dsldesc}
    True if at least one member other than \cc{member} is selected.
\end{dsldesc}

\begin{dslsig}
containsOnly(member: Any): Logical
\end{dslsig}
\begin{dsldesc}
    True if \cc{member} is the only selected member.
\end{dsldesc}

\begin{dslsig}
sum(func: Member => Integer): Integer
\end{dslsig}
\begin{dsldesc}
    Sum of values obtained by applying \cc{func} on each selected member.

    Typically used to sum the value of some knowledge field of the selected components.
    Scala's placeholder underscore is useful here:

\begin{lstlisting}[style=snippet]
val swarm = subsetOf(robots)
constraint { swarm.sum(_.arms) > 7 }
\end{lstlisting}
\end{dsldesc}

\begin{dslsig}
all(func: Member => Logical): Logical
\end{dslsig}
\begin{dsldesc}
    True if predicate \cc{func} holds for all selected members.
\end{dsldesc}

\begin{dslsig}
some(func: Member => Logical): Logical
\end{dslsig}
\begin{dsldesc}
    True if predicate \cc{func} holds for at least one selected member.
\end{dsldesc}

\begin{dslsig}
allEqual(func: Member => Any): Logical
\end{dslsig}
\begin{dsldesc}
    True if result of \cc{func} is the same for every selected member. In other
    words, the set of values yielded by \cc{func} for each selected member has at most
    one element.

    Typically used to ensure that components selected for a role all have the same value
    of some knowledge field, e.g., belong to the same team, have the same rank, etc.
\end{dsldesc}

\begin{dslsig}
allDifferent(func: Member => Any): Logical
\end{dslsig}
\begin{dsldesc}
    True if result of \cc{func} is different for every selected member. In other
    words, the set of values yielded by \cc{func} for each selected member is the
    same size as the set of selected members.

    Typically used to ensure that components selected for a role all differ in some
    knowledge field, e.g., no two components from the same team are picked.
\end{dsldesc}

\begin{dslsig}
disjointAfterMap[T,M](funcThis: Member => T, \\
~~~~~~~~~~~~~~~~~~~~~~other: MemberGroup[M], \\
~~~~~~~~~~~~~~~~~~~~~~funcOther: M => T): Logical
\end{dslsig}
\begin{dsldesc}
    True if, after converting the selected members of this and the other group to a
    common type \cc{T}, the resulting sets are disjoint.

    There are two types of usage for this function. One type is ensuring that two groups
    of same or similar types of components are partitioned according to some variable
    --- e.g., groups of workers disjoint over the projects they are working on.

    The other type is ensuring that two groups of different types of components do not
    mix with regard to some property. An example of this would be a sort of ``wolf,
    goat, cabbage'' scenario: the \cc{eaters} role in a \cc{Shore}
    ensemble must be disjoint with the \cc{foods} role, so that none of the
    eaters will eat any of the food.
\end{dsldesc}

%%%

\subsection{Collections of Member Groups}

Special behavior is defined for collections of \cc{MemberGroup}s, which are typically
obtained by mapping a collection of ensembles to one of their roles. The purpose of this
behavior is to support a common idiom: ensuring that membership of a role in multiple
ensembles does not overlap.

The following example assigns workers to teams in a way that no worker is assigned to
two teams:

\begin{lstlisting}[style=ensembles]
val workers: List[Worker] = /* ... */

class Team(val id: Int) extends Ensemble {
  val teamMembers = subsetOf(workers, _ > 0)
}

val teams = rules {
    for (i <- 1 to 4) yield new Team(i)
}

constraint { teams.map(_.teamMembers).allDisjoint }
constraint { teams.map(_.teamMembers).cardinality === workers.size }
\end{lstlisting}

\noindent
A collection of member groups has the following methods:

\begin{dslsig}
cardinality: Integer
\end{dslsig}
\begin{dsldesc}
    Cardinality of the collection, i.e., the total number of selected members across all
    groups in the collection.
\end{dsldesc}

\begin{dslsig}
allDisjoint: Logical
\end{dslsig}
\begin{dsldesc}
    True if all groups in the collection are disjoint, i.e., no member is selected in
    more than one group.
\end{dsldesc}

%%%

\subsection{\texttt{Policy} class}

The \cc{Policy} class represents the security policy in a single object, and provides
methods to initiate solving and examine its results.

The \cc{resolve()} method is the most straightforward way to interact with the policy
object:

\begin{dslsig}
resolve(): Boolean \\
resolve(limit: Long): Boolean
\end{dslsig}
\begin{dsldesc}
    Find a valid solution and record security actions. If no utility expression is
    defined, returns the first solution that satisfies all constraints. If an utility
    expression is defined, returns the maximum utility solution, or the best solution
    that could be found within a time limit. Returns \cc{true} if a solution was found,
    or \cc{false} if not.

    When \cc{limit} is specified, it is used as a time limit for the solving process, in
    milliseconds. If the method does not return before the time limit expires, the best
    solution found so far is recorded.

    This is an all-in-one method that performs all solving steps automatically. To
    customize the solving process, it is necessary to use the methods below.
\end{dsldesc}

\bigskip
\noindent
The following attributes are available as soon as a solution is attempted:

\begin{dslsig}
exists: Boolean
\end{dslsig}
\begin{dsldesc}
    True if a solution was found.
\end{dsldesc}

\begin{dslsig}
instance: Ensemble
\end{dslsig}
\begin{dsldesc}
    Reference to an instance of the root ensemble class. Through \cc{instance}, it is
    possible to examine role and sub-ensemble assignment.
\end{dsldesc}

\begin{dslsig}
actions: Iterable[Action]
\end{dslsig}
\begin{dsldesc}
    Generated list of security actions, collected from all sub-ensembles. Contains
    objects of type \cc{AllowAction}, \cc{DenyAction} and \cc{NotifyAction}.
\end{dsldesc}

\begin{dslsig}
solutionUtility: Int
\end{dslsig}
\begin{dsldesc}
    Total utility of the solution, if one exists. If the solution exists but has no
    utility expressions, this will return zero.
\end{dsldesc}

\begin{dslsig}
allows(actor: Component,\\
~~~~~~~action: String,\\
~~~~~~~subject: Component): Boolean
\end{dslsig}
\begin{dsldesc}
    Return true if \cc{actor} is permitted to \cc{action} on \cc{subject}.

    See section~\ref{dsl:c:access} for semantics of access control queries.
\end{dsldesc}

\bigskip
\noindent
In case a customization of the solving process is needed, the following methods are
available to run the solving step-by-step:

\begin{dslsig}
init() \\
init(limit: Long)
\end{dslsig}
\begin{dsldesc}
    Reset the solver, configure time limit, delete all solutions and recorded security
    actions, and prepare for finding a solution.

    Time limit is in milliseconds and applies across all subsequent solving runs. If
    the time limit expires while a call to \cc{solve()} is in progress, the solving
    process stops and \cc{solve()} returns \cc{false}. See section~\ref{dsl:c:time} for
    details.

    Must be called whenever the situation changes, otherwise the policy will reflect the
    previous state of ensembles, components, and the environment. In particular, must be
    called before the first call to \cc{solve()}.
\end{dsldesc}

\begin{dslsig}
solve(): Boolean
\end{dslsig}
\begin{dsldesc}
    Find one solution. Return \cc{true} if a valid solution is found, \cc{false}
    otherwise.
    
    This method can be called repeatedly to iterate over all solutions. If no solution
    is found, and a previously-found solution exists, it will still be accessible.
    
    If no utility expression is specified, repeated calls will yield successive
    solutions. With a utility expression, each successive call will find a solution with
    higher utility than the previous one. If no such solution exists, \cc{solve()} will
    return \cc{false}, even if other solutions exist with equal utility. To find the
    solution with maximum utility, the following idiom is used:\nopagebreak
\begin{lstlisting}[style=snippet]
while (policy.solve()) {}
\end{lstlisting}
\end{dsldesc}

\begin{dslsig}
commit()
\end{dslsig}
\begin{dsldesc}
    Commit current solution, generate security rules, and send notifications to
    components. Must be called before accessing \cc{actions} for a new solution.
\end{dsldesc}
