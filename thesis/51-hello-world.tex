\section{Hello, World!}
\label{dsl:hello}

The following code snippet describes a simple ensemble with one member --- the
\lstinline{greeter}, who is granted the \lstinline{"greet"} permission for each of
\lstinline{people}.

\begin{lstlisting}[
    label=lst:hello,
    style=ensembles,
]
class Person(name: String) extends Component { |\label{lst:hello:component}|
  name(name)
}

class SimpleScenario(val people: Seq[Person]) { |\label{lst:hello:model}|

  class HelloWorld extends Ensemble { |\label{lst:hello:ensemble}|
    val greeter = oneOf(people) |\label{lst:hello:role}|

    allow(greeter, "greet", people) |\label{lst:hello:grant}|
  }

  val policy = Policy.root(new HelloWorld) |\label{lst:hello:root}|
}
\end{lstlisting}

Line~\ref{lst:hello:component} defines a very simple component --- a person with a name.

Line~\ref{lst:hello:model} defines a scenario class. We use scenario classes to enclose
the ensemble definitions and related data. In this case, it's the list of
\lstinline{people} on which our ensemble will be operating.

Line~\ref{lst:hello:ensemble} defines a root ensemble, and line~\ref{lst:hello:role}
specifies a \lstinline{greeter} role, which is \lstinline{oneOf} the \lstinline{people}
in this scenario.

With line~\ref{lst:hello:grant} we grant the selected \lstinline{greeter} the permission
to perform action \lstinline{"greet"} on any of the \lstinline{people}. Actions are
specified as strings.

Finally, line~\ref{lst:hello:root} sets up a new instance of the \lstinline{HelloWorld}
ensemble as a root. This tells the solver where to start.

\bigskip

We have specified our scenario, but the code above doesn't actually \textit{do} anything.
We need a way to execute it and provide the list of people. Let's create a companion
object:

\begin{lstlisting}[
    label=lst:hello-runner,
    style=ensembles,
    firstnumber=15,
]
object SimpleScenario {
  val Names = Seq("Roland", "Lilith", "Mordecai", "Brick")

  def main(args: Array[String]): Unit = {
    val people = for (name <- Names) yield new Person(name)
    val scenario = new SimpleScenario(people) |\label{lst:hello-runner:instance}|

    scenario.policy.resolve() |\label{lst:hello-runner:resolve}|
    for (action <- scenario.policy.actions) println(action) |\label{lst:hello-runner:output}|
  }
}
\end{lstlisting}

The \lstinline{main} function generates a list of \lstinline{Person} instances, which is
then used to instantiate the scenario on line~\ref{lst:hello-runner:instance}. The
\lstinline{resolve} call on line~\ref{lst:hello-runner:resolve} instructs the framework
to find and apply the first solution. Line~\ref{lst:hello-runner:output} prints out all
permission grants.

When the program is executed, its output will look like this:

\begin{lstlisting}[style=output]
AllowAction(<Component:Roland>,greet,<Component:Roland>)
AllowAction(<Component:Roland>,greet,<Component:Lilith>)
AllowAction(<Component:Roland>,greet,<Component:Mordecai>)
AllowAction(<Component:Roland>,greet,<Component:Brick>)
\end{lstlisting}

We can see that the solver selected the first \lstinline{Person} to be a greeter,
and granted them permission to perform the \lstinline{"greet"} action on all the other
\lstinline{Person}s.
