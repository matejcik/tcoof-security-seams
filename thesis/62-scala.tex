\section{From Scala to DSL}
\label{impl:scala}

The Scala language has many features which make it suitable as a host language for an
internal DSL. The ability to use blocks of code as expressions allows us to define
custom language constructs. By-name parameters enable delayed evaluation, a necessary
feature for writing declarative code in otherwise mostly imperative language. Arbitrary
nesting, together with code in class bodies, makes the resulting language flow feel
natural. Scala's type system allows us to retain advantages of strong typing and build
generic containers that do the right thing --- and at the same time, type inference is
powerful enough so that users of the DSL almost never come into contact with type
annotations. Finally, operator overloading lets us emulate primitive types with custom
classes, and implicit conversions ensure that the emulated types interoperate with the
built-in ones.

This section describes the overall architecture of the framework, and technical details
of all language constructs used in the DSL. Although we give a brief overview of each
language feature that we are using, this section is mostly intended for readers already
familiar with the Scala language.

%%%

\subsection{DSL Constructs}
\label{impl:scala:verbs}

The DSL uses a number of distinct language constructs. An ensemble definition is, in
Scala terms, implementing a child class:

\begin{lstlisting}[style=snippet]
class MyEnsemble extends Ensemble { /* ... */ }
\end{lstlisting}

Creating a dynamic role is realized as creating a member field from the result of a
function call:

\begin{lstlisting}[style=snippet]
val role = subsetOf(members)
\end{lstlisting}

The call runs directly in the class body. In Scala, this means that it is executed as
part of the constructor.

Scala does have first-class functions, but it does not have \textit{top-level}
functions. Each function must be defined within a class. Functions that are available at
ensemble definition time are in fact methods of the class \cc{Ensemble}. Their results
are tied to the enclosing \cc{Ensemble} instance, which is also where the registration
actually happens; if the result of \cc{subsetOf} in the above snippet was not saved to a
member variable, the selection from \cc{members} would still happen behind the scenes.

This is also one of the reasons ensemble instances must be explicitly registered with
\cc{rules} or \cc{ensembles}: without the registration call, the enclosing instance
would not know about the nested instances. While it would technically be possible to
examine the class through reflection and find all \cc{Ensemble} instances in member
attributes, this would be needlessly fragile in practice.

\medskip

Apart from ``functions'', the DSL also makes use of ``constructs''. The following code
looks distinctly different from the above:

\begin{lstlisting}[style=snippet]
utility {
    val occupied = occupants.size + assignees.cardinality
    occupied * occupied
}
\end{lstlisting}

Although \cc{utility} looks more like a keyword or a flow-control statement, it is in
fact a normal method. Its definition could look like this:

\begin{lstlisting}[style=snippet, language=Scala]
def utility(util: Integer)
\end{lstlisting}

I.e., it is a method that takes a single argument of type \cc{Integer}. The special
syntax is enabled by two useful properties of Scala.

One, code blocks are expressions whose value is the last expression in the block. In
this case, the value of the code block is \dop{occupied * occupied}. Given that a code
block is an expression like any other, it is possible to use code blocks in unusual
places, such as function arguments. The following is a contrived but perfectly valid
example of Scala code:

\begin{lstlisting}[style=ensembles, language=Scala]
val maximum = math.max(
  {
    val a = someFunc()
    if (a > 5) a
    else {
        val b = someOtherFunc()
        a + b
    }
  }, 0)
\end{lstlisting}

Two, Scala allows omitting parentheses in certain conditions. In particular, when
calling a function with one argument, it is possible to omit the enclosing parentheses
if the argument is a code block enclosed in curly braces. The following three statements
are equivalent:

\begin{lstlisting}[style=snippet]
utility(3)
utility({ 3 })
utility { 3 }
\end{lstlisting}

This way it is possible to define methods that can behave as custom language constructs.
Note that this feature is particularly useful in combination with by-name parameters.
See section~\ref{impl:scala:byname} for detailed explanation.


%%%

\subsection{Ensemble Structure}
\label{impl:scala:ensemble}

Each \cc{Ensemble} instance contains the following data:
\begin{itemize}
    \item Collection of role objects, representing dynamically assigned roles
    \item Collection of \cc{EnsembleGroup} objects for each \cc{rules} or \cc{ensembles}
    call
    \item Collection of constraints, in the form of callables that return a \cc{Logical}
    \item Collection of associated security actions and notifications
    \item Situation predicate, if defined
    \item Utility function, if defined
\end{itemize}

For each type of data, methods are available for configuring this data at ensemble
definition time.

As the total number of methods is rather large, and the related functionalities are
mostly independent, each is implemented in a separate trait. We maintain separation of
concerns between traits and declare their dependencies through inheritance and self-type
annotations.

Trait inheritance works the same as normal class inheritance. Most of ensemble traits
inherit from \cc{Initializable}, allowing them to override the \cc{_init} method and
access the solver instance; see section~\ref{impl:scala:initialization} for details.
Inheritance also allows us to inject implicit conversions from \cc{CommonImplicits}
trait.

Self-type annotation in a trait enforces that any concrete class implementing that trait
must also conform to the declared self-type. This usually means that it must implement
some other trait. For example, the following self-type in trait \cc{WithRoles} informs
the compiler of a dependency on two other traits:
\begin{lstlisting}[style=snippet]
trait WithRoles {
  this: WithConstraints with WithSelectionStatus =>
  /* ... */
}
\end{lstlisting}

The \cc{WithRoles} trait is then allowed to access members of the other traits on self.

Functional difference between inheritance and self-type annotation is subtle, but the
takeaway is that self-types define a looser coupling, because they do not lock-in the
inheritance hierarchy. We use self-types to declare that a trait \textit{uses} a
functionality from another trait, but is not an extended version of it.

\medskip

The following list describes the individual traits of \cc{Ensemble}.

\newenvironment{trait}[1]%
    {%
        \par\vspace{0.6em}
        {\bfseries\ttfamily\raggedright\large#1}
        \setlength{\parindent}{0em}
        \setlength{\parskip}{0.3em}
        \begin{adjustwidth}{1cm}{}
    }%
    {%
        \end{adjustwidth}
    }%

\pagebreak

\begin{trait}{Initializable}
    Provides an initialization hook, plus access to the solver instance through the
    \cc{_solverModel} property.
\end{trait}

\begin{trait}{CommonImplicits}
    Defines implicit conversion from \cc{Int} to \cc{Integer}, from \cc{Boolean} to
    \cc{Logical}, and adds methods to collections of \cc{MemberGroup}s. This makes those
    implicit conversions available in ensemble definition context.
\end{trait}

\begin{trait}{WithName}
    Provides a \cc{name} property. Mainly for debugging purposes.
\end{trait}

\begin{trait}{WithActions}
    Provides \cc{allow}, \cc{deny} and \cc{notify} functions, and a method to collect
    results of these functions from sub-ensembles. To achieve that, it has a dependency
    on \cc{WithEnsembleGroups}.
\end{trait}

\begin{trait}{WithConstraints}
    Provides the \cc{constraint} function, storage of registered constraints, and
    functionality related to converting constraints to solver objects.
\end{trait}

\begin{trait}{WithEnsembleGroups}
    Provides the ability to register groups of sub-ensembles, and the \cc{rules} and
    \cc{ensembles} methods. Because groups must activate themselves based on the parent
    ensemble selection status, this trait requires \cc{WithSelectionStatus}.
\end{trait}

\begin{trait}{WithRoles}
    Provides the ability to register roles via role functions. Because roles implicitly
    use constraints, and the \cc{subsetOf} function allows adding a constraint on the
    subset cardinality, this requires the \cc{WithConstraints} trait. In addition,
    \cc{WithSelectionStatus} is required to support group activation based on ensemble
    selection status.
\end{trait}

\begin{trait}{WithSelectionStatus}
    Provides a \cc{BoolVar} indicating the ensemble's selection status.
\end{trait}

\begin{trait}{WithUtility}
    Provides methods to define and query the utility expression of the ensemble, and
    collects utility value from sub-ensembles. To access sub-ensembles, this trait
    requires the \cc{WithEnsembleGroups} trait.
\end{trait}

\medskip

The \cc{Ensemble} class itself defines the \cc{situation} function and handling of the
situation predicate. While it would be straightforward to extract this functionality to
a separate trait, we opted to keep it in \cc{Ensemble}. The functionality is small
enough that it does not clutter the class, esp. given that \cc{Ensemble} is otherwise
empty; and it does not figure in the dependency graph, so extracting it would have
little benefit for the rest of the code (unlike \cc{WithSelectionStatus}, which is also
small and uncomplicated, but is a dependency of two other traits).

%%%

\subsection{Implicit Conversions}

Scala's implicit conversions allow values of one type to be automatically converted to a
different type, if they appear in a context where the target type is required. Two kinds
of implicit conversions are available. With \cc{implicit def}, a function returning the
target type is applied to the value in question. An \cc{implicit class} creates an
instance of a new class, effectively allowing to add new methods to existing types.

Code that uses implicit conversions is less readable, because there are non-local hidden
calls. In the following example, an implicit conversion adds a method named \cc{foo} to
a value of type \cc{Int}. It is impossible to know what is \cc{foo} just from this
snippet; one would need to locate all methods named \cc{foo} in the codebase and then
figure out which implicit conversion is in play.
\begin{lstlisting}[style=snippet]
val x: Int = 7
val y = x.foo()
\end{lstlisting}
\noindent
For this reason, it is usually better to avoid implicit conversions in normal code.
However, they can be extremely helpful when creating a DSL.

For an implicit conversion to take effect, its function or class must be in scope:
either defined in the same class (or a parent class), or explicitly imported.

\medskip

We use implicit conversions for several different purposes. The most common one is
upgrading \cc{Int}s to \cc{Integer}s and \cc{Boolean}s to \cc{Logical}s. This is
important for seamless interoperability of native integer or boolean expressions with
solver constraints, as described in section~\ref{impl:solver:constraints}. The
appropriate conversion functions are defined in trait \cc{CommonImplicits}, which is
mixed in where required. Importantly, it is mixed into the main \cc{Ensemble} class,
thus ensuring that user code will have these conversions in scope whenever defining an
ensemble.

\medskip

Implicit conversions also help cut down on the number of different method overloads. The
method \cc{unionOf} accepts either an iterable of \cc{MemberGroup}s, or a variable
number of \cc{MemberGroup} arguments. However, we want to support using \cc{Component}s
for \cc{unionOf} directly. Otherwise, every time the user wanted to make an union of a
role and a fixed list of components, they would need to specify a role for the component
list. Scala has no support for union types, so it would be impossible to specify a
function that takes a variable number of ``\cc{MemberGroup} or \cc{Component} or
\cc{Iterable[Component]}'' arguments.

Instead, an implicit conversion from \cc{Component} or \cc{Iterable[Component]} to
\cc{MemberGroup[Component]} is defined, allowing the users to use either roles or fixed
sets of components interchangeably.

Similarly, the \cc{allow}, \cc{deny} and \cc{notify} verbs should accept components,
list of components, or roles as their arguments. In the \cc{notify} case, it is simple
enough to provide three overloads for three acceptable argument types. However,
\cc{allow} and \cc{deny} have two arguments of this type. A full set of overloads would
need 9 variants of each verb. Instead, each of these verbs only accepts
\cc{MemberGroup}s as arguments, and implicit conversions ensure that components and
lists of components are usable as well.

\medskip

Two distinct implicit conversions are used with ensemble collections. First, an
\cc{EnsembleGroup} converts to a list of all its members, so that methods like \cc{map}
can be used on results of \cc{rules} call directly. Second, an implicit class
\cc{WithMembersIterable} adds methods \cc{cardinality} and \cc{allDisjoint} to
collections of \cc{MemberGroup}s. This enables the following common idiom:
\begin{lstlisting}[style=snippet]
val r = rules(/* list of ensembles */)
constraint { r.map(_.someRole).allDisjoint }
\end{lstlisting}
\noindent
The variable \cc{r} of type \cc{EnsembleGroup} is converted to \cc{Iterable[E <:
Ensemble]}, so that \cc{map} can be applied on it. This iterable is a plain sequence of
objects of the appropriate ensemble type, which has a member role \cc{someRole}. The
result of the map is therefore a plain sequence of \cc{MemberGroup} objects.

This sequence is then implicitly converted to a class \cc{WithMembersIterable}, which
has the method \cc{allDisjoint}, ensuring that no member is selected more than once for
\cc{someRole}.

Note that the implicit conversion turns the \cc{EnsembleGroup} into a list of
\textit{all} its members, not just the selected ones. This is because at the time the
conversion is used, the list of selected members is not yet known.

The provided methods work due to the fact that inactive (not selected) ensembles do not
have any members. Empty roles add zero to \cc{cardinality}, and are guaranteed to be
disjoint with non-empty ones.

An implicit conversion from an \cc{EnsembleGroup} to its \cc{allMembers} carries some
risk of confusion when used in an inappropriate context. E.g., when inspecting a
solution, the user might inadvertently invoke the implicit conversion and their code
will visit even ensembles that are not part of the solution. We consider this risk
acceptable because, again, ensembles that are not selected do not have any members.

%%%

\subsection{Operator Overloading}
\label{impl:scala:operators}

Method names in Scala are allowed to use special characters, and operators are defined
as regular methods with the appropriate name. For instance, overloading the \dop{+}
operator on an \cc{IntegerInt} type could look like this:
\begin{lstlisting}[style=snippet]
def +(other: Integer): Integer = other match {
    case IntegerInt(value) => new IntegerInt(value + this.value)
    // ...
}
\end{lstlisting}
\noindent
Combined with implicit conversions, both of the following will work:

\begin{lstlisting}[style=snippet]
val leftHand: Integer = someInteger + 5
val rightHand: Integer = 5 + someInteger
\end{lstlisting}
\noindent
The first case is simple. The expression \dop{someInteger + 5} is interpreted as
\dop{someInteger.+(5)}. Because \cc{someInteger} has a \dop{+} operator, Scala tries to
use it. The operator is defined for an \cc{Integer}, and there is an implicit conversion
available that upgrades \cc{5} to \cc{Integer}. The resulting instance is passed to the
\dop{+} method.

The second case is slightly more complicated. As before, the expression translates to
\dop{5.+(someInteger)}. In this case, the built-in \cc{Int} type does not have an
appropriate \dop{+} operator for handling anything \cc{someInteger} could be converted
to. It is necessary to search the list of possible implicit conversions of the left-hand
operand, to see if one of them comes with an appropriate definition of \dop{+}.

\medskip

This process works fine for most operators, but fails for \dop{==}. The problem here is
that \dop{==} is defined on the root class \cc{Any} with the following signature:

\begin{lstlisting}[style=snippet]
final def ==(arg: Any): Boolean
\end{lstlisting}
\noindent
The method is final, so it cannot be overridden. It calls the \cc{equals} method
internally, which is available for overriding, but then the return value is
\cc{Boolean}, which is not appropriate when we need the result as a \cc{Logical}.

It is possible to overload a specialized version of \dop{==} with a different return
type. For the \cc{Integer} trait, the following definitions are desirable:

\begin{lstlisting}[style=snippet]
def ==(other: Integer): Logical
def ==(other: Int): Logical
\end{lstlisting}
\noindent
We need to have a special variant for \cc{Int}, because implicit conversion won't help
us here: \dop{someInteger.==(5)} can use the default implementation for type \cc{Int}.

The problem with this is that the integration is not seamless. If the \cc{Int} value
appears on the right-hand side, it will always be able to use the built-in \dop{==}.
This is made worse by the fact that both of the following will compile:

\begin{lstlisting}[style=snippet]
val l1: Logical = someInteger == 5
val l2: Logical = 5 == someInteger
\end{lstlisting}
\noindent
The type of \dop{5 == someInteger} is \cc{Boolean}, which has an implicit conversion to
\cc{Logical}, and so it is valid in every context that requires a \cc{Logical} type. But
the result of the comparison is always false, because we are comparing unequal types.

\medskip

The \cc{Integer} trait uses a triple-equals \dop{===} operator for comparing equality in
the desired way. This is a convention used by many other Scala-based DSLs and language
customizations. Scala compiler will emit a warning whenever two unrelated types are
compared with \dop{==}, which should notify the user that they made a mistake.

We considered raising an exception in an overridden \cc{equals} method, but that would
silence this warning, and would not work when \cc{Int} is the left-hand operand anyway.

%%%

\subsection{Type Bounds}

One of the features required from the DSL is seamless working with different ensemble
and component types. Consider the following code:
\begin{lstlisting}[style=ensembles]
class Worker(val rank: String) extends Component
val workers: Iterable[Worker] = /* ... */
class Root extends Ensemble {
    val supervisor = oneOf(workers)
    constraint { supervisor.all(_.rank == "supervisor") }
    allow(supervisor, "supervise", workers)
}
\end{lstlisting}

The \dop{supervisor.all()} call must accept a function whose argument is of type
\cc{Worker} --- otherwise, it would be impossible to access the \cc{rank} attribute.
That means that \cc{supervisor} must carry the information that its member type is
\cc{Worker}, not just a generic \cc{Component}. At the same time, the \cc{allow} call
must accept \cc{supervisor} as its argument.

\pagebreak
Roles are of type \cc{MemberGroup}, which is generic over its member type. The signature
looks like this:
\begin{lstlisting}[style=snippet]
class MemberGroup[+MemberType]
\end{lstlisting}

The \dop{+} symbol indicates that the class is \textit{covariant} in the \cc{MemberType}
argument. That means that for superclass \cc{A} and its subclass \cc{B},
\cc{MemberType[B]} is considered a subclass of \cc{MemberType[A]}.

Method \cc{MemberGroup.all} accepts functions of type \cc{MemberType => Boolean}. In our
example, member type is \cc{Worker}.

Method \cc{accept} takes an argument of type \cc{MemberGroup[Component]}, so it accepts
a \cc{MemberGroup[Worker]} object.

The type signature of \cc{oneOf} looks like this:
\begin{lstlisting}[style=snippet]
def oneOf[C <: Component](items: Iterable[C]): MemberGroup[C]
\end{lstlisting}
The method takes an iterable of objects of concrete type \cc{C}, which is a subtype of
\cc{Component}, as indicated by the \dop{<:} symbol. It returns a \cc{MemberGroup}
parameterized by the appropriate concrete type. Scala's type inference makes sure that
when passing an iterable of \cc{Worker}s, the returned \cc{MemberGroup} will have
\cc{Worker} as its member type.

\medskip

Ensemble groups work in a similar way, except that \cc{EnsembleGroup} is a subclass of
\cc{MemberGroup} specialized so that its member type must be a subtype of \cc{Ensemble}.
This is the signature:
\begin{lstlisting}[style=snippet]
class EnsembleGroup[+EnsembleType <: Ensemble]
    extends MemberGroup[EnsembleType]
\end{lstlisting}

The \dop{+} symbol indicates type covariance: an \cc{EnsembleGroup} of a concrete
type of ensemble is a subclass of \cc{EnsembleGroup[Ensemble]}. The \dop{<:} symbol
indicates a type bound: the member type must be a subtype of \cc{Ensemble}.

Similar to role functions, \cc{rules} and \cc{ensembles} must be defined with
appropriate type signatures, so that they return the appropriate variant of
\cc{EnsembleGroup}.

%%%

\subsection{Initialization}
\label{impl:scala:initialization}

Every ensemble and group object requires an instance of the solver, through which
variables and constraints are constructed. This is not available at ensemble definition
time.

In terms of Scala, the whole policy is a series of instances of nested classes. In order
to propagate a solver object, the user of the DSL would be required to add parameters to
their ensemble class definitions and manually propagate them when instantiating
sub-ensembles. This violates separation of concerns: the solver object is an
implementation detail of the framework, not something the users should care or even know
about.

Instead, the ensemble tree is constructed without access to the solver, and the solver
object is passed to it in a separate initialization step. The \cc{Initializable} trait
defines a method \cc{_init}, which allows classes and traits to hook into the
initialization process. The \cc{Policy} object calls \cc{_init} on the root ensemble,
and the \cc{Ensemble} class is responsible for propagating the call to its ensemble
groups and role objects. Ensemble groups are in turn responsible for propagating the
call to sub-ensembles. This way it is ensured that every part of the policy tree is
initialized.

The initialization runs in three phases:
\begin{enumerate}
  \item \cc{ConfigPropagation} propagates a \cc{Config} object, which contains the newly
  created instance of the solver.
  \item \cc{VarsCreation} is the phase where solver variable objects are created. The
  class \cc{MemberGroup} initializes its set and activation variables, and \cc{Ensemble}
  initializes its selection variable.
  \item \cc{RulesCreation} can use variables created in the previous step to generate
  and post solver constraints.
\end{enumerate}

While it might be possible to perform the initialization in a single pass, it would make
it more difficult to implement properly. Developers would need to be extra careful about
initialization order, e.g., make sure that an initialization step is not using variables
from a child (or parent) object whose initialization isn't finished yet. Multi-phase
initialization removes this source of fragility. Phase (1) can be done by every object
individually. Phase (2) only requires that the solver is available, i.e., that phase (1)
has finished everywhere, and phase (3) can rely on the fact that \textit{all} variables
across all objects have been generated in the previous phase.

\medskip

Several component traits of \cc{Ensemble} implement their own \cc{_init} steps. This is
possible because Scala has a well-defined trait linearization order, and calling
\cc{super._init()} within a trait will invoke the \cc{_init} implementation in the next
trait up.

%%%

\subsection{By-Name Parameters}
\label{impl:scala:byname}

As stated in section~\ref{impl:scala:initialization}, the solver object is not available
at ensemble definition time --- or, more precisely, at ensemble instantiation time. Most
of constraint definition and role creation is done in class body, and this code actually
runs when the corresponding instance is constructed. This poses two related but distinct
problems.

First, consider a role that relies on external data:

\begin{lstlisting}[style=ensembles]
class Scenario(val workers: mutable.ArrayBuffer[Worker]) {
  class Root extends Ensemble {
    val role = subsetOf(workers)
    constraint { role.cardinality > 5 }
  }
  val policy = Policy.root(new Root)
}

object Scenario {
  def main() {
    val scenario = new Scenario(/* ... */)
    scenario.policy.resolve()
    scenario.workers.append(new Worker(/* ... */))
    scenario.policy.resolve()
  }
}
\end{lstlisting}

One would expect the second \cc{resolve} call to use the newly added \cc{Worker}
instance. However, at that point, the ensemble seems to already exist, and the old value
of \cc{workers} was used to construct \cc{role}. Groups register their members at
construction time (see section~\ref{impl:solver:groups}), so the original argument no
longer affects the set of role members.

It would be useful if the framework could somehow ``refresh'' the policy definition
using new data.

Second, the \cc{constraint} statement works with \cc{role.cardinality}. We expect the
result to be a \cc{Logical} object referencing an underlying \cc{BoolVar}. But as stated
above, at this point the solver instance is not available, so a \cc{BoolVar} cannot be
created.

Both of those issues are solved using \textit{by-name parameters}.

\medskip

In Scala, functions are first-class objects, so naturally it is possible to pass them as
arguments explicitly. That is what happens in this statement:
\begin{lstlisting}[style=snippet]
val role = subsetOf(members, _ > 1)
\end{lstlisting}
The expression \dop{_ > 1} is a shortcut for \dop{x => x > 1}. This is a function of
type \dop{Integer => Logical}, i.e., takes an \cc{Integer} argument and returns a
\cc{Logical}.

By-name parameters are basically parameters that are passed as functions implicitly. The
signature of a \cc{constraint} verb is:
\begin{lstlisting}[style=snippet, language=Scala]
def constraint(clause: => Logical): Unit
\end{lstlisting}
\cc{clause} is a by-name parameter, and its type is ``expression of type \cc{Logical}''.
The important point is that unlike normal parameters, the expression is not evaluated
until actually used in the \cc{constraint} method body. Furthermore, the following
works:
\begin{lstlisting}[style=snippet]
val clauseFun: () => Logical = clause _
\end{lstlisting}
Using the above syntax, the ``expression'' parameter is converted to a normal function
with no arguments. The \cc{constraint} method stores all such functions in a collection
and only runs them in the \cc{RulesCreation} initialization phase, when the solver is
available and all variables are already created.

The verbs \cc{situation} and \cc{utility} also take by-name parameters and call them
at appropriate times.

\medskip

Section~\ref{dsl:c:root} specifies that the root ensemble must be instantiated in the
\cc{Policy.root} method. The reason is that the argument of this method is actually a
by-name parameter too. The policy object does not store the instantiated policy tree, it
stores a \textit{builder} function which constructs it. On every call to \cc{init()} or
\cc{resolve()}, the policy tree is instantiated from scratch, and so it reflects the
current values of all external variables.

%%%

\subsection{Variadic arguments}

Many functions that accept iterables also accept variadic arguments. For example, it is
possible to call \cc{oneOf} in two ways:
\begin{lstlisting}[style=snippet]
val a = oneOf(listOfMembers)
val b = oneOf(memberA, memberB, memberC)
\end{lstlisting}

Scala natively supports variadic arguments. For the following two signatures, the type
of \cc{items} is the same:
\begin{lstlisting}[style=snippet, language=Scala]
def oneOf(items: Seq[Component])
def oneOf(items: Component*)
\end{lstlisting}

One minor issue with the second signature is that it allows the list of arguments to be
empty. This does not cause any problems in practice --- after all, the user could as
well pass an empty list of components --- but still, it would be cleaner to disallow
code like \dop{val x = oneOf()}.

To achieve this, all variadic functions actually have two arguments: one is mandatory,
the other is variadic. This is the definition of \cc{oneOf}:
\begin{lstlisting}[style=ensembles, language=Scala]
def oneOf[C <: Component](itemFirst: C, itemRest: C*): Role[C] =
    oneOf(itemFirst +: itemRest)
\end{lstlisting}
Type of \cc{itemFirst} is \cc{C}, and type of \cc{itemRest} is \cc{Seq[C]}. The code
prepends the first item to the list and calls the other overload of \cc{oneOf}. It is
possible to call \cc{oneOf} with one or more arguments; calling with no arguments is
illegal.
