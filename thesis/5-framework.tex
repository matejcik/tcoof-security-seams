\chapter{Ensemble Framework}
\label{dsl}

We have designed and developed a framework for managing access control with ensembles.
It consists of a \textit{domain-specific language} (DSL) for describing the ensembles,
and a \textit{runtime environment} which can analyze the ensemble description and
generate the specified ensembles from available components.

The DSL is internal within the Scala language. That means that it uses the basic syntax
of Scala, is compiled with the Scala compiler and runs on the Java virtual machine. We
have implemented a number of functions and language constructs that enable the user to
express ensemble-related concepts in a succinct and readable way. We also leverage
Scala's strong type system to enforce typing checks and catch problems at compile-time.

This chapter serves as a user guide to the framework. Section~\ref{dsl:hello} shows
a~very simple example of usage. Section~\ref{dsl:concepts} introduces each of the main
concepts in the framework, their function and semantics. Section~\ref{dsl:example}
translates our running example from chapter~\ref{running-example} to the DSL. Finally,
section~\ref{dsl:reference} lists all features available in the DSL and their usage.

\section{Typographical Conventions}

In the following text, \textit{italic type} is used to emphasize newly introduced
terms. After the term is explained, its further occurrences are set in normal type.

To highlight source code elements, such as functions, types, variable names, and code
snippets, we use \cc{monospace font}.

\section{Overview}
\label{dsl:overview}

The purpose of the framework is to analyze the DSL-specified ensemble structure and
assign components to appropriate roles in a way that fits all the constraints, and/or
maximizes values of some variables. This can be understood as a constraint satisfaction
problem (CSP). Therefore, the main component of the framework is a CSP solver. We use
the word \textit{solving} for the process of determining ensemble membership, and a
\textit{solution} is a particular assignment of components to roles in the ensembles.
Only when the scenario is solved, we can generate access control rules based on the
solution.

\medskip

Input of the framework consists of two parts. First is a description of the ensemble
structure, expressed with the DSL. Second is a collection of components, which are
supposed to be assigned to roles.

\medskip

When the computation is done, the framework outputs a reference to the solution. It is
possible to examine which ensembles were activated and which components were selected
for which roles. The framework also emits access control rules and notifications.

\section{Hello, World!}
\label{dsl:hello}

The following code snippet describes a simple ensemble with one member --- the
\lstinline{greeter}, who is granted the \lstinline{"greet"} permission for each of
\lstinline{people}.

\begin{lstlisting}[
    label=lst:hello,
    style=ensembles,
]
class Person(name: String) extends Component { |\label{lst:hello:component}|
  name(name)
}

class SimpleScenario(val people: Seq[Person]) { |\label{lst:hello:model}|

  class HelloWorld extends Ensemble { |\label{lst:hello:ensemble}|
    val greeter = oneOf(people) |\label{lst:hello:role}|

    allow(greeter, "greet", people) |\label{lst:hello:grant}|
  }

  val policy = Policy.root(new HelloWorld) |\label{lst:hello:root}|
}
\end{lstlisting}

Line~\ref{lst:hello:component} defines a very simple component --- a person with a name.

Line~\ref{lst:hello:model} defines a scenario class. We use scenario classes to enclose
the ensemble definitions and related data. In this case, it's the list of
\lstinline{people} on which our ensemble will be operating.

Line~\ref{lst:hello:ensemble} defines a root ensemble, and line~\ref{lst:hello:role}
specifies a \lstinline{greeter} role, which is \lstinline{oneOf} the \lstinline{people}
in this scenario.

With line~\ref{lst:hello:grant} we grant the selected \lstinline{greeter} the permission
to perform action \lstinline{"greet"} on any of the \lstinline{people}. Actions are
specified as strings.

Finally, line~\ref{lst:hello:root} sets up a new instance of the \lstinline{HelloWorld}
ensemble as a root. This tells the solver where to start.

\bigskip

We have specified our scenario, but the code above doesn't actually \textit{do} anything.
We need a way to execute it and provide the list of people. Let's create a companion
object:

\begin{lstlisting}[
    label=lst:hello-runner,
    style=ensembles,
    firstnumber=15,
]
object SimpleScenario {
  val Names = Seq("Roland", "Lilith", "Mordecai", "Brick")

  def main(args: Array[String]): Unit = {
    val people = for (name <- Names) yield new Person(name)
    val scenario = new SimpleScenario(people) |\label{lst:hello-runner:instance}|

    scenario.policy.resolve() |\label{lst:hello-runner:resolve}|
    for (action <- scenario.policy.actions) println(action) |\label{lst:hello-runner:output}|
  }
}
\end{lstlisting}

The \lstinline{main} function generates a list of \lstinline{Person} instances, which is
then used to instantiate the scenario on line~\ref{lst:hello-runner:instance}. The
\lstinline{resolve} call on line~\ref{lst:hello-runner:resolve} instructs the framework
to find and apply the first solution. Line~\ref{lst:hello-runner:output} prints out all
permission grants.

When the program is executed, its output will look like this:

\begin{lstlisting}[style=output]
AllowAction(<Component:Roland>,greet,<Component:Roland>)
AllowAction(<Component:Roland>,greet,<Component:Lilith>)
AllowAction(<Component:Roland>,greet,<Component:Mordecai>)
AllowAction(<Component:Roland>,greet,<Component:Brick>)
\end{lstlisting}

We can see that the solver selected the first \lstinline{Person} to be a greeter,
and granted them permission to perform the \lstinline{"greet"} action on all the other
\lstinline{Person}s.

\section{Core Concepts}
\label{dsl:concepts}

%%%

\subsection{Components}
\label{dsl:c:components}

Every entity that the runtime knows about needs to be represented as an object derived
from \cc{Component}. That means connected devices, locks, and even people, are treated
as components. They can be members of ensembles and inhabit roles.

A component represents the system's knowledge of an entity, but the framework has no
control over it. From its point of view, a component and its data are purely inputs.
This is also how we can represent people as components: as far as the runtime is
concerned, they are view-only.

Component instances don't need any methods, and in fact they should not have any. The
possible exception to this recommendation is ``computed knowledge''. Sometimes it is
useful to have a helper method that returns the result of some computation over the
component's attributes. For example, a worker component can have knowledge of its
location, and a method \cc{isInLunchRoom} that returns true if the location is a
lunchroom. The result of this method is still component knowledge, i.e., something we
know about the component, but does not need to be stored as a value.

Components should come from outside the ensemble definitions, preferably from outside
the scenario class. Component instances should never be created within an ensemble. All
knowledge about a component should be represented in its attributes.

%%%

\subsection{Ensembles}
\label{dsl:c:ensembles}

Every ensemble is represented as an instance of a subclass of \cc{Ensemble}. The
instance holds references to roles and their assigned components, sub-ensembles,
situation predicates and constraints.

If there is just one ensemble or sub-ensemble for a particular purpose, it can be
specified as a singleton with Scala's \cc{object} keyword. If more ensembles of the same
type are needed, a class is more appropriate. All instances of the class must be
explicitly created in the parent ensemble body. Every instance must also be registered,
using either \cc{rules} or \cc{ensembles} call:

\begin{lstlisting}[
    label=lst:rules,
    style=ensembles,
]
class Root extends Ensemble {
  object SubEnsembleObject extends Ensemble {
    // ...
  }
  class SubEnsemble(id: Int) extends Ensemble {
    // ...
  }

  val subEnsembles = for (i <- 1 to 5) yield new SubEnsemble(i)
  rules(subEnsembles)
  ensembles(SubEnsembleObject)
}
\end{lstlisting}

Section~\ref{dsl:c:situations} explains the difference between \cc{rules} and
\cc{ensembles}.

The class body can specify role membership, sub-ensembles, constraints, and situation
predicates. Each of these features is described in its own subsection.

%%%

\subsection{Ensemble Activation and Situations}
\label{dsl:c:situations}

An ensemble can be active or inactive. We also use the term \textit{selected} --- as in,
the ensemble is selected as a member of the parent ensemble. Ensembles registered with
the \cc{rules} function are active by default. When registered with \cc{ensembles}, the
framework dynamically determines whether the ensemble should be active or inactive,
based on constraints in its parents. 

An active ensemble takes part in the computation and all constraints specified in the
ensemble must be satisfied. Inactive ensembles are not considered, their roles have no
members, and all their sub-ensembles are also inactive --- i.e., an ensemble can only be
active if all its parents are active.

It is possible to specify a boolean \textit{situation predicate}. If the predicate
evaluates to false, the ensemble and all of its children are deactivated. Note that an
ensemble registered with \cc{ensembles} can still be inactive if its situation predicate
is true; it is a necessary condition, not a sufficient one.

The situation predicate is specified via the \cc{situation} construct. At most one
situation predicate can be used per ensemble; if more than one \cc{situation} is
defined, the last one is used and all others are ignored.

%%%

\subsection{Roles}
\label{dsl:c:roles}

Fundamentally, an ensemble is a collection of components assigned to distinct roles.

Some role assignments can be known in advance. Such roles don't need to be explicitly
created. It is a good practice to assign the component to a member variable of the
ensemble, but it is perfectly legal to use a variable from an outer scope directly.

Other roles are assigned dynamically by the framework. For these, it is necessary to
describe parameters of the assignment and a collection of candidates.

The function \cc{subsetOf} specifies that a particular role is a subset of a given
collection of components, or a subset of inhabitants of a different role. An optional
argument allows in-line specification of a constraint on the subset's cardinality, e.g.,
that no more than $N$ components must be selected.

As a shortcut, the \cc{oneOf} function defines that exactly one of the candidates will
be selected for the role. The \cc{allOf} function defines that all of the candidates
will be selected, effectively converting components directly to roles. This can be
useful in some cases, but it is usually not necessary to do it explicitly.

The function \cc{unionOf} can link several roles into a single group. This is useful in
situations where a constraint applies over many roles together and it would be difficult
to express it in terms of the individual roles.

Dynamically assigned roles are sets of components. A component can be a member of
multiple roles, but cannot be a member of the same role in the same ensemble more than
once.

The result of each of these functions is a \textit{role object}. After a solution is
found, this object can be used to access a list of the selected elements. At solving
time, however, the results are not yet available, and instead specialized methods of the
role object must be used. See section~\ref{dsl:r:membergroups} for a complete list of
available operations.

%%%

\subsection{Constraints}
\label{dsl:c:constraints}

A \textit{constraint} places arbitrary limits on the solution. Unlike situation
predicates, which are evaluated beforehand, constraints are applied during the search
for a solution. It is thus a good place to specify requirements that the solution must
fulfill. Constraints can be used to specify that membership of some role must be
disjoint with another role, mark one role's cardinality as strictly smaller than that of
another role, etc.

Constraint predicates are specified with a \cc{constraint} call. There can be multiple
\cc{constraint} calls in an ensemble, and all specified constraints must be fulfilled in
the solution. If a constraint in an active ensemble cannot be fulfilled, no solution
will be found.

Constraints in inactive ensembles are ignored.

Care must be taken when creating constraints over the results of \cc{subsetOf} and the
other role functions. Role objects define several methods, such as \cc{all}, \cc{sum},
and others, whose results are valid constraint objects. However, if the list of selected
members is accessed directly while specifying a constraint, an exception will be thrown.
See section~\ref{dsl:r:membergroups} for details.

%%%

\subsection{Solution Utility}
\label{dsl:c:utility}

It is possible to attach a utility expression to an ensemble. If present, the framework
will by default try to maximize the value of this expression when looking for solutions.
In other words, solutions with higher utility are preferred.

Utility expression can be specified with the \cc{utility} construct. Only one utility
expression can be specified per ensemble. If more than one \cc{utility} is used, only
the last one takes effect.

If multiple ensembles specify an utility expression, the total utility of the solution
is a simple sum of all the individual utilities. This is usually the desired behavior.
E.g., when calculating utilities for individual lunchrooms from the running example,
each room has its own utility, and we are interested in the sum of all rooms.

Sometimes this solution is insufficient, however. Maybe the total utility of an ensemble
is the average utility of each of its sub-ensembles. If that's the case, \cc{utility}
should not be used in the sub-ensembles, only in the parent. The sub-ensembles can
instead specify a method calculating the partial utility, which will then be used in
parent's utility expression:

\begin{lstlisting}[
    label=lst:subutility,
    style=ensembles,
]
class Root extends Ensemble {
  class SubEnsemble extends Ensemble {
    val someRole = subsetOf(members, _ < 4)
    def subUtility = someRole.cardinality * someRole.sum(_.weight)
  }

  val subEnsembles = for (_ <- 1 to 5) yield new SubEnsemble
  utility {
    subEnsembles.map(_.subUtility).reduce(_ + _) / subEnsembles.size
  }
}
\end{lstlisting}

Only active ensembles are counted towards the utility total.

Similar to role objects, utilities are not actual numbers. Methods like \cc{sum} are not
available and more elaborate calculations must be built from basic operators.

%%%

\subsection{Root Ensemble}
\label{dsl:c:root}

One ensemble must be designated as top-level, or root, in the scenario. This tells the
framework where to start with solving. The root ensemble is always active, and
\cc{situation} specified in the root has no effect.

It would be natural to specify the root ensemble as an \cc{object}, but due to
limitations of the DSL embedding, that is not possible. The root ensemble \textbf{must}
be a class, instantiated in a call to \cc{Policy.root}:

\begin{lstlisting}[
    label=lst:rootens,
    style=ensembles,
]
class Example {
  class Root extends Ensemble {
    // ...
  }
  val policy = Policy.root(new Root)
}
\end{lstlisting}

See section~\ref{impl:scala:byname} for explanation of this requirement.

Result of the \cc{Policy.root} call is a \cc{Policy} object. By convention, we always
store it in a member variable named \cc{policy} in a scenario class.

%%%

\subsection{Scenarios and Solving}
\label{dsl:c:solving}

Once a policy is specified, it is necessary to supply the actual \cc{Component}
instances and other contextual data. These are usually enclosed in a so-called
\textit{scenario class}.

In terms of code, the scenario class is a plain class and does not need to implement any
particular traits. Its purpose is code organization: it is a good practice to keep the
policy definition (i.e., the hierarchy of \cc{Ensemble}s) in the same scope with the
required context information. Having the context encapsulated within a class enables
use-cases such as multiple instances of the same policy for different buildings.

By convention, a \textit{policy instance} is stored in an attribute named \cc{policy}.

The policy instance, created with \cc{Policy.root} call, provides access to the
framework's solving functionality. The most basic feature is the \cc{resolve()} method,
which performs the computation, assigns members to roles, and generates access control
rules. The solution can be examined through the \cc{instance} member, which is a
reference to the instance of the root ensemble. The rules can be accessed through the
\cc{actions} member or queried via the \cc{allows()} method.

The framework does not monitor the situation continuously. The solution generated with
\cc{resolve()} is valid for the situation at the time it was called. If the situation
changes, component knowledge updates, etc., the user of the framework is responsible for
calling \cc{resolve()} again.

Every call to \cc{resolve()} runs from scratch. Previously established ensembles are
dissolved and members are assigned to roles without consideration for previous
assignments. This matches the operation of ensemble-based systems: ensembles are loose
coalitions that form and re-form based on current conditions. For handling of
persistence, refer to subsection~\ref{dsl:c:notifications}.

If no utility functions are specified, \cc{resolve()} will find and apply the first
solution that satisfies all the constraints. When utility functions are present, it will
find and apply a solution with the highest utility.

It is also possible to iterate over possible solutions manually. The \cc{init()} method
will reset solver state. After that, every call to \cc{solve()} will find a new
solution, viewable through \cc{instance}, or return \cc{false} if no more solutions are
found. Finally, a call to \cc{commit()} will apply access control rules from the current
solution.

The behavior is slightly different with utility functions; there is an added constraint
that each new solution must have higher utility than the one before it. That means that
it will not be possible to examine every valid solution, and the computation will stop
at the first solution that cannot be improved. Of course, finding out that a particular
solution cannot be improved might take a long time.

%%%

\subsection{Time Limits}
\label{dsl:c:time}

Some security policies can have an exponential number of solutions. Absent other
constraints, there are $2^N$ possible results of \cc{subsetOf} on $N$ members. Finding a
solution with maximum utility could therefore take a very long time.

To manage this issue, it is possible to configure a time limit after which the search
stops. As explained in the previous section, an optimizing solver iterates through valid
solutions, searching for those with increasing utility. By setting a time limit, we are
effectively declaring that we are interested not in the optimal solution, but the best
that can be found inside the limit.

Time limits are most useful in scenarios with utility expressions. They provide little
benefit when iterating over all valid solutions, as the search can usually be stopped
simply by not requesting another solution.

An optional parameter to \cc{init()} or \cc{resolve()} specifies a time limit in
milliseconds. The countdown starts at the time the method is called, and carries across
calls to \cc{solve()}. If the time limit expires while \cc{solve()} is running, it will
stop the computation and return false, and all subsequent calls will also return false
--- same as if no more solutions can be found.

%%%

\subsection{Access Control}
\label{dsl:c:access}

When the scenario is solved and component assignments are known, the runtime emits
specified access control rules. There are two available functions: \cc{allow} and
\cc{deny}. Both take three arguments: actor, action and subject. Actors and subjects can
be components, collections of components, or roles. Action must be a string.

Specifying a collection of actors or subjects is the same as specifying each actor and
each subject one by one. Specifying a role applies the rule to selected members of that
role.

If the ensemble is active, access control directives will be emitted. The framework
takes a default-deny approach. If a triplet does not exist in emitted directives, the
permission is denied. If an \cc{allow} triplet exists, the permission is granted, unless
a \cc{deny} triplet also exists. This way, it is possible to grant a~wide permission in
an ensemble, but refine it in a sub-ensemble or a different situation-specific ensemble.

%%%

\subsection{Notifications and Persistence}
\label{dsl:c:notifications}

Using the function \cc{notify}, it is possible to attach messages to components. This is
the only way the framework can affect components (which are usually autonomous entities
beyond our control). The notification feature serves two purposes.

First, it is possible to query the notifications inside an ensemble. This way it is
possible to persist earlier configurations. Consider this example:

\begin{lstlisting}[
    label=lst:notify,
    style=ensembles,
]
case class Reservation(room: Room) extends Notification

class SeatReservations(room: Room) extends Ensemble {
  val alreadyReserved = workers.filter(_.notified(Reservation(room)))
  val newlyReserved = oneOf(workers.filter(_.askingForReservation))

  allow(alreadyReserved, "enter", room)
  allow(newlyReserved, "enter", room)
  notify(newlyReserved, Reservation(room))
}
\end{lstlisting}

Workers that have reserved seats in previous runs will still be granted the \cc{"enter"}
permission on subsequent runs. If we didn't attach the notification, they would lose the
permission when the solution is rerun.

Second, a notification action is recorded as part of the generated access control rules.
Users of the system can listen for these notification actions and forward them to
components. This can be useful to, e.g., send a message to a worker's smartphone, to
inform them about their seat reservation.

Notification messages must implement trait \cc{Notification}. It is useful to define
them as case classes, so that it is possible to filter them by value, as demonstrated in
the example.


\section{Implementing the Running Example}
\label{dsl:example}

%%%

\subsection{Overview}

Our scenario consists of workers, which are assigned to projects; workrooms, which are
also assigned to projects; and lunchrooms, which are unassigned. Workers, workrooms, and
lunchrooms will be represented as components. Two distinct sub-problems exist, each with
its own parameters: assignment of workrooms and assignment of lunchrooms. These can be
naturally described as separate ensembles.

\medskip

When the building is open, workers are allowed to enter all workrooms assigned to their
project. We will create an ensemble for every project, and this ensemble will have the
following roles:
\begin{itemize}
    \renewcommand\labelitemi{--}
    \setlength\itemsep{0em}
    \item \textit{project workers}, inhabited by all workers for that project
    \item \textit{project rooms}, inhabited by all workrooms for that project
\end{itemize}
The ensemble will grant all project workers access to all project rooms.

\medskip

Lunchrooms open at lunch time. Workers can indicate that they are hungry, which we
represent as a knowledge field on the worker component. We would like to collect hungry
workers in small ensembles, each granting access to a single lunchroom. To accomplish
that, we will create an ensemble for every lunchroom, with the following roles:

\begin{itemize}
    \renewcommand\labelitemi{--}
    \setlength\itemsep{0em}
    \item \textit{occupants}, inhabited by all workers currently in the room, plus all
    workers that have previously been assigned to the room
    \item \textit{assignees}, inhabited by hungry workers who are not yet assigned
\end{itemize}
The ensemble will attempt to collect assignees from the pool of all hungry workers, up
until room capacity is filled. An additional constraint is that every member of this
ensemble must be assigned to the same project. That means that if there are existing
occupants, new assignees must have the same project as them. If the room has no current
occupants, new assignees can be selected from any project.

In addition, not all seatings are equally good. We want to use lunchrooms sparingly:
given the choice between putting a worker into an empty lunchroom and an occupied one,
the occupied should be picked, so that we keep the empty lunchroom available for other
projects.

Once a satisfactory solution has been found, the ensemble will allow both occupants and
assignees to enter the lunchroom, and notify new assignees that a seat was found for
them.

%%%

\subsection{Implementation}

Following this description is a full listing of the policy for the running example,
including definitions of components.

Lines~\ref{ex:components:start}--\ref{ex:components:end} define component types. All
rooms are of common supertype \cc{Room}. Apart from name, the \cc{LunchRoom} has a
knowledge field \cc{capacity}, stating its maximum occupancy.

\cc{Worker}s have three knowledge fields: their assigned \cc{project} (specified in
constructor), their \cc{hungry} status, and their current \cc{location}. A helper method
\cc{isInLunchRoom} returns \cc{true} if the worker's location is a lunchroom, as an
example of ``computed knowledge'' from section~\ref{dsl:c:components}

Next, case classes are defined on lines~\ref{ex:casecls:start}--\ref{ex:casecls:end}.
Projects do not need to be components, but we still need structured information about
them, particularly their list of assigned workrooms. We also create a case class for
lunchroom assignment notifications.

Scenario class definition starts at line~\ref{ex:scenario}. Its inputs are lists of
projects, workers, workrooms, and lunchrooms. An instance of this scenario class could
represent a workday in a single building. Opening and closing times are part of the
security policy and are hard-coded at lines~\ref{ex:times:start}--\ref{ex:times:end}.

Line~\ref{ex:now} defines a variable to represent current time. In a real-life
deployment, this might be represented by a reference to system clock; however, for our
testing, it is useful to set the current time explicitly.

We also define a helper attribute \cc{workersByProject} for easier access to lists of
workers from the same project.

Class \cc{RoomAssignment} at line~\ref{ex:root} represents the policy root. It defines
one pseudo-role, \cc{hungryWorkers}, which is inhabited by all workers who are (1)
hungry, (2) not currently in a lunchroom, and (3) not already assigned to a lunchroom.
This will be the pool from which the lunchroom ensembles select candidates. Furthermore,
the root ensemble contains definitions of the two sub-ensembles for our two
sub-problems.

Workroom ensemble is described by the class \cc{WorkroomAssignment} on
line~\ref{ex:workroom}. It takes a project definition as a parameter. Its situation
predicate specifies that it is only active between \cc{BuildingOpenTime} and
\cc{BuildingCloseTime}. The statically-assigned role \cc{projectWorkers} is simply the
list of workers for the ensemble's project. The grant at line~\ref{ex:workroomgrant}
allows all \cc{projectWorkers} to enter all \cc{workrooms} of the project.

\medskip

Lunchroom ensemble is represented by the \cc{LunchroomAssignment} class at
line~\ref{ex:lunchroom}, and takes a lunchroom as an argument. Like the workroom
ensemble, it also has a situation predicate; this time specifying that it applies
between \cc{LunchOpenTime} and \cc{LunchCloseTime}.

The first role, \cc{occupants} at line~\ref{ex:occupants}, is a statically determined
list of workers that are either (a) already assigned to the room, as determined by the
appropriate \cc{LunchRoomAssigned} notification, or (b) physically present in the room,
as indicated by their \cc{location} attribute.

We use \cc{occupants} to calculate \cc{freeSpaces}, the number of remaining seats in the
room. Then, at line~\ref{ex:assignees}, we define the \cc{assignees} role as a
dynamically selected subset of \cc{hungryWorkers}, limiting its size to the number of
free seats.

The \cc{eaters} role at line~\ref{ex:eaters} is a union of \cc{occupants} and
\cc{assignees}. It is not a meaningful role in the ensemble, but implementation-wise, it
is needed to specify the constraint on the next line: all \cc{eaters} must have the same
value of their \cc{project} attribute.

Lines~\ref{ex:utility:start}--\ref{ex:utility:end} define the utility expression. Fuller
rooms are preferred, which is expressed by setting the utility to the square of total
number of used seats. This way, a solution that places two workers in the same lunchroom
is measured as better than a solution that places each of them in a separate room.

Finally, \cc{assignees} are notified of their assignment at line~\ref{ex:notify},
and line~\ref{ex:lunchroomgrant} allows all \cc{eaters} to enter the room.

\medskip

Back at the \cc{RoomAssignment} level, lines~\ref{ex:rules:start}--\ref{ex:rules:end}
create instances of the sub-ensembles. An instance of \cc{WorkroomAssignment} is
generated for every project, and an instance \cc{LunchroomAssignment} is generated for
every lunchroom. The \cc{rules} call configures the sub-ensembles to be selected
whenever their situation predicate is true, i.e., they are always active in their
specified time-frames.

The constraint at line~\ref{ex:globalconstraint} ensures that all instances of the
\cc{assignees} role are disjoint, or in other words, that no worker can get a seat
reservation in more than one lunchroom at the same time.

Line~\ref{ex:policy} instantiates the policy object and makes it available as an
attribute.

%%%
\pagebreak

\subsection{Source Code}

\begin{lstlisting}[style=ensembles]
// Different types of rooms |\label{ex:components:start}|
abstract class Room(name: String) extends Component {
  name(s"Room:$name")
}
class LunchRoom(name: String, val capacity: Int)
  extends Room("Lunch" + name)
class WorkRoom(name: String)
  extends Room("Work" + name)

// Worker assigned to a project, can be hungry or not
class Worker(id: Int, val project: Project) extends Component {
  name(s"Worker:$id:${project.name}")
  var hungry = false
  var location: Option[Room] = None

  def isInLunchRoom: Boolean =
    location.map(_.isInstanceOf[LunchRoom]).getOrElse(false)
} |\label{ex:components:end}|

// Project with pre-assigned workrooms |\label{ex:casecls:start}|
case class Project(name: String, workrooms: Seq[WorkRoom])

// Notification for lunchroom assignment
case class LunchRoomAssigned(room: LunchRoom) extends Notification |\label{ex:casecls:end}|

class LunchScenario(val projects: Seq[Project],  |\label{ex:scenario}|
                    val workers: Seq[Worker],
                    val workrooms: Seq[WorkRoom],
                    val lunchrooms: Seq[LunchRoom]) {
  // Opening times of the building and of the lunchrooms
  val BuildingOpenTime  = LocalTime.of( 7, 30)  |\label{ex:times:start}|
  val BuildingCloseTime = LocalTime.of(21,  0)
  val LunchOpenTime     = LocalTime.of(11, 30)
  val LunchCloseTime    = LocalTime.of(15,  0)  |\label{ex:times:end}|

  val DefaultNow = LocalTime.of(8, 42)
  var now = DefaultNow  |\label{ex:now}|
  
  // mapping projects to lists of workers
  val workersByProject = workers.groupBy(_.project)

  class RoomAssignment extends Ensemble { |\label{ex:root}|
    name("assign workers to projects and rooms")

    // list of all hungry workers waiting for a lunchroom
    val hungryWorkers = workers.filter { w =>
      w.hungry &&
      !w.isInLunchRoom &&
      !w.notified[LunchRoomAssigned]
    }

    // Each worker assigned to a project can access all workrooms
    // assigned to that project when the building is open.
    class WorkroomAssignment(project: Project) extends Ensemble { |\label{ex:workroom}|
      name(s"assign workrooms to workers on project ${project.name}")

      situation { (now isAfter BuildingOpenTime) &&
                  (now isBefore BuildingCloseTime) }

      val projectWorkers = workersByProject.getOrElse(project, Seq.empty)
      allow(projectWorkers, "enter", project.workrooms) |\label{ex:workroomgrant}|
    }

    // Each hungry worker will get an assigned lunchroom so that
    // no lunchroom is over capacity and workers from different
    // projects do not meet in the same lunchroom.
    class LunchroomAssignment(room: LunchRoom) extends Ensemble { |\label{ex:lunchroom}|
      name(s"assign workers to lunchroom ${room.name}")

      // Only activate when lunchrooms are open
      situation { (now isAfter LunchOpenTime) &&
                  (now isBefore LunchCloseTime) }

      // list of previously assigned workers
      val occupants = workers.filter { w =>  |\label{ex:occupants}|
        w.notified(LunchRoomAssigned(room)) |\textbar\textbar|
        w.location.contains(room)
      }

      // newly-assigned hungry workers must fit into free space
      val freeSpaces = room.capacity - occupants.size
      val assignees = subsetOf(hungryWorkers, _ <= freeSpaces) |\label{ex:assignees}|

      val eaters = unionOf(occupants, assignees) |\label{ex:eaters}|
      constraint { eaters.allEqual(_.project) }

      // Set the solution utility to square of the number of occupants,
      // i.e., prefer many workers in one room over few workers in many rooms
      utility { |\label{ex:utility:start}|
        val occupied = assignees.cardinality + occupants.size
        occupied * occupied
      } |\label{ex:utility:end}|

      // grant access rights and notify newly selected hungry workers
      notify(assignees, LunchRoomAssigned(room)) |\label{ex:notify}|
      allow(eaters, "enter", room) |\label{ex:lunchroomgrant}|
    }

    val workroomAssignments =  |\label{ex:rules:start}|
      rules(projects.map(new WorkroomAssignment(_)))
    val lunchroomAssignments =
      rules(lunchrooms.map(new LunchroomAssignment(_))) |\label{ex:rules:end}|

    // ensure that a worker is not assigned to more than one lunchroom
    constraint(lunchroomAssignments.map(_.assignees).allDisjoint) |\label{ex:globalconstraint}|
  }

  val policy = Policy.root(new RoomAssignment) |\label{ex:policy}|
}
\end{lstlisting}
\pagebreak

\section{Reference}
\label{dsl:reference}

Scala is a strongly typed language, so a Scala-internal DSL is also strongly typed. In
order to make this section more concise, however, we will be using simplified type
signatures. The following simplifications are used:

\begin{itemize}
\item Whenever a function has bounded type parameters, we omit them in favor of the most
general type.
\item If a function does not return a value, the return type \cc{Unit} is omitted.
\item Whenever a function has a variadic argument (denoted with an asterisk), at least
one argument must be provided.
\item For every function with a variadic argument of type \cc{T}, denoted as \dop{T*},
an overload exists that takes a single \cc{Iterable[T]} argument instead.
\item We use type name \cc{Role} for brevity, but that type does not exist. The actual
type is \cc{MemberGroup[Component]}.
\end{itemize}

Unabridged function signatures are available in the Scaladoc API documentation in the
\cc{apidoc} directory of the accompanying archive.

\medskip

For readers not deeply familiar with Scala, we point out the \textit{by-name parameters}
feature. Whenever an argument type is prefixed with \dop{=>}, it is not evaluated
immediately, but only when it is used. This allows us to write expressions that would
fail at ensemble definition time, but work fine when the framework executes them.

%%%

\newenvironment{dslsig}%
    {%
        \par\vspace{0.6em}\bfseries\ttfamily\raggedright
    }%
    {%
        \vspace{-0.2em}
    }%

\newenvironment{dsldesc}%
    {%
        \nopagebreak
        \setlength{\parindent}{0em}
        \setlength{\parskip}{0.3em}
        \begin{adjustwidth}{1cm}{}
    }%
    {%
        \end{adjustwidth}
    }%

%%%

\subsection{\texttt{Component} class}
\label{dsl:r:component}

\cc{Component} must be used as a superclass of every component type. Only one function
is available at declaration time:

\begin{dslsig}
name(nm: String)
\end{dslsig}
\begin{dsldesc}
    Set a name of the component. This is useful for debugging purposes, when printing
    out ensemble memberships.
\end{dsldesc}

\medskip

\noindent
Component instances also have several methods from trait \cc{Notifiable} for querying
received notifications:

\begin{dslsig}
notifications: Iterable[Notification]
\end{dslsig}
\begin{dsldesc}
    Return a collection of all \cc{Notification} instances received by this component.
\end{dsldesc}

\begin{dslsig}
notified(notification: Notification): Boolean
\end{dslsig}
\begin{dsldesc}
    Query a specific notification. Return \cc{true} if the exact specified notification
    was received by this component.
\end{dsldesc}

\begin{dslsig}
notified[N <: Notification]: Boolean
\end{dslsig}
\begin{dsldesc}
    Query a notification class. Return \cc{true} if any notification of type \cc{N} was
    received by this component.
\end{dsldesc}

%%%

\subsection{\texttt{Integer} type}

\cc{Integer} is a generic reference to an integer number whose value might not be known
until a solution is found. It is usually a result of operations on member group objects.
Basic arithmetic operators on \cc{Integer}s are overloaded to return \cc{Integer}s and
basic comparison operators are overloaded to return \cc{Logical}s. Implicit conversion
from \cc{Int} is available, so that it is possible to mix \cc{Integer} calculations with
standard Scala math.

One notable imperfection is that the equality operator \dop{==} cannot be overloaded in
Scala. For comparing values of \cc{Integer}s, use the triple-equals \dop{===} operator
instead. The operator \dop{==} will compare object identities and return a~boolean.

%%%

\subsection{\texttt{Logical} type}

\cc{Logical} is a generic reference to a truth value which might not be known until a
solution is found. It is usually the type of constraint operations. Basic boolean
operators on \cc{Logical}s are overloaded to return \cc{Logical}s. Implicit conversion
from \cc{Boolean} is available, so that it is possible to mix \cc{Logical} expressions
with statically evaluated booleans.

%%%

\subsection{\texttt{Ensemble} class}

The security policy consists of a nested series of classes deriving from \cc{Ensemble}.
Most of the ensemble definition happens in the body of \cc{Ensemble}, so this class
provides most of the available functions.

\begin{dslsig}
name(nm: String)
\end{dslsig}
\begin{dsldesc}
    Set a descriptive name of the ensemble. This is useful for code documentation and 
    for debugging purposes, when printing out ensemble memberships.
\end{dsldesc}

\begin{dslsig}
utility(util: => Integer)
\end{dslsig}
\begin{dsldesc}
    Assign an utility function to the ensemble. \cc{util} is an \cc{Integer} expression
    that is evaluated for each solution being tested. Refer to
    subsection~\ref{dsl:c:utility} for detailed semantics.
\end{dsldesc}

\begin{dslsig}
rules(ensembles: Ensemble*): EnsembleGroup
\end{dslsig}
\begin{dsldesc}
    Register sub-ensemble(s) with static activation. Sub-ensembles registered via this
    function \textit{must} be activated if possible.

    Each sub-ensemble must be registered with \cc{rules} or \cc{ensembles} to take part
    in the computation.
\end{dsldesc}

\begin{dslsig}
ensembles(ensembles: Ensemble*): EnsembleGroup
\end{dslsig}
\begin{dsldesc}
    Register sub-ensemble(s) with dynamic activation. Sub-ensembles that are registered
    via this function \textit{can} be activated by the solver, if that leads to a good
    solution.

    Each sub-ensemble must be registered with \cc{rules} or \cc{ensembles} to take
    part in the computation.
\end{dsldesc}

\pagebreak
\begin{dslsig}
oneOf(items: Component*): Role \\
oneOf(role: Role): Role
\end{dslsig}
\begin{dsldesc}
    Define a role inhabited by \textit{exactly one} of the specified components or
    inhabitants of the specified role.
\end{dsldesc}

\begin{dslsig}
allOf(items: Component*): Role
\end{dslsig}
\begin{dsldesc}
    Define a role inhabited by \textit{all} of the specified components.

    This function is useful for explicit conversion of components to role objects.
    However, in most cases, components and collections of components are implicitly
    converted to roles as needed. E.g., the following two ensemble definitions are
    equivalent:
\begin{lstlisting}[style=ensembles]
val members = for (_ <- 1 to 5) yield new Member

object WithAllOf extends Ensemble {
    val role = allOf(members)
    allow(role, "open", door)
}

object WithoutAllOf extends Ensemble {
    allow(members, "open", door)
}
\end{lstlisting}
\end{dsldesc}

\begin{dslsig}
subsetOf(items: Component*): Role \\
subsetOf(role: Role): Role \\
subsetOf(role: Role, cardinality: Integer => Logical): Role
\end{dslsig}
\begin{dsldesc}
    Define a role inhabited by a \textit{subset} of the specified components or
    inhabitants of the specified role.
    
    The optional argument \cc{cardinality} specifies a constraint on the subset's
    cardinality. It is a function that takes an \cc{Integer} argument, representing the
    subset's cardinality, and returns a \cc{Logical} result configuring whether the
    cardinality is valid. It is possible to use Scala's placeholder underscore as a
    shortcut, i.e.:

\begin{lstlisting}[style=snippet]
val role = subsetOf(components, _ < 10)
\end{lstlisting}
\end{dsldesc}

\begin{dslsig}
unionOf(roles: Role*): Role
\end{dslsig}
\begin{dsldesc}
    Defines a role whose members are a \textit{union of specified roles}. Specifically,
    any component that inhabits one of the \cc{roles} also inhabits the union role, and
    if a component inhabits the union role, then there exists at least one role in
    \cc{roles} which the component also inhabits. This is mainly useful for specifying
    constraints over collections of roles that would otherwise be difficult to express
    individually; e.g., total size of the union must not exceed a specified number.
\end{dsldesc}

\begin{dslsig}
allow(actors: Role, action: String, subjects: Role)
\end{dslsig}
\begin{dsldesc}
    Grant permission to each inhabitant of \cc{actors} role to perform \cc{action} on
    each inhabitant of the \cc{subjects} role.

    Through implicit conversions, it is possible to use a component or an iterable of
    components in place of any of the \cc{Role} arguments.
\end{dsldesc}

\pagebreak
\begin{dslsig}
deny(actors: Role, action: String, subjects: Role)
\end{dslsig}
\begin{dsldesc}
    Deny permission to each inhabitant of \cc{actors} role to perform \cc{action} on
    each inhabitant of the \cc{subjects} role.

    Through implicit conversions, it is possible to use a component or an iterable of
    components in place of any of the \cc{Role} arguments.
\end{dsldesc}

\begin{dslsig}
notify(targets: Role, message: Notification)
\end{dslsig}
\begin{dsldesc}
    Send a \cc{message} to each of \cc{targets}. The message is persisted
    across solver runs, and its presence can be queried when forming ensembles. See
    subsection~\ref{dsl:c:notifications} for detailed semantics and
    subsection~\ref{dsl:r:component} for query methods.

    Through implicit conversions, it is possible to use a component or an iterable of
    components in place of the \cc{Role} argument.
\end{dsldesc}

\begin{dslsig}
constraint(clause: => Logical)
\end{dslsig}
\begin{dsldesc}
    Set up a constraint that must be satisfied in every solution. The clause must be of
    type \cc{Logical}, because it is propagated to the constraint programming
    engine, where it limits the search space. Therefore, it is possible to use role
    object expressions as constraints.

    Multiple constraints can be specified in an ensemble and each one must be satisfied
    in a valid solution.
\end{dsldesc}

\begin{dslsig}
situation(predicate: => Boolean)
\end{dslsig}
\begin{dsldesc}
    Set up a situation predicate. The predicate is evaluated \textit{before} the solver
    starts processing the ensemble. If it evaluates to \cc{false}, the ensemble
    is excluded from the solution.

    \cc{situation} has no effect in the root ensemble.
\end{dsldesc}

%%%

\subsection{Member Groups}
\label{dsl:r:membergroups}

The base class \cc{MemberGroup[T]} maintains a collection of components or ensembles. A
so-called ``role object'' is in fact a \cc{MemberGroup[C <: Component]}. Ensemble groups
are represented by a subclass \cc{EnsembleGroup}, which performs additional handling
related to ensemble hierarchies. However, all functionality relevant to the DSL is
defined in the base class, and thus identical for roles and ensemble groups.

For simplicity, we will use the type name \dop{Member} to stand in for the member type
of a \cc{MemberGroup}.

The member group provides a notion of \textit{selected members} --- a subset of the
member collection which is considered part of the solution. For instance, a role created
with the \cc{oneOf} function will have exactly one selected member.

All \cc{Integer} methods can be used to build constraints with arithmetic and comparison
operators. All \cc{Logical} methods can be used as constraints directly, or combined
with other constraints using boolean operators. For simplicity, the descriptions use
terms like ``true if'' or ``false if'', but keep in mind that the truth value of
\cc{Logical} is only relative to a candidate solution.

\begin{dslsig}
selectedMembers: Iterable[Member]
\end{dslsig}
\begin{dsldesc}
    List of \cc{Member} instances that are selected for the solution.

    Throws an exception if no solution has been generated.
\end{dsldesc}

\begin{dslsig}
cardinality: Integer
\end{dslsig}
\begin{dsldesc}
    Cardinality of the group, i.e., the number of selected members.
\end{dsldesc}

\begin{dslsig}
contains(member: Any): Logical
\end{dslsig}
\begin{dsldesc}
    True if the specified member is selected.
\end{dsldesc}

\begin{dslsig}
containsOtherThan(member: Any): Logical
\end{dslsig}
\begin{dsldesc}
    True if at least one member other than \cc{member} is selected.
\end{dsldesc}

\begin{dslsig}
containsOnly(member: Any): Logical
\end{dslsig}
\begin{dsldesc}
    True if \cc{member} is the only selected member.
\end{dsldesc}

\begin{dslsig}
sum(func: Member => Integer): Integer
\end{dslsig}
\begin{dsldesc}
    Sum of values obtained by applying \cc{func} on each selected member.

    Typically used to sum the value of some knowledge field of the selected components.
    Scala's placeholder underscore is useful here:

\begin{lstlisting}[style=snippet]
val swarm = subsetOf(robots)
constraint { swarm.sum(_.arms) > 7 }
\end{lstlisting}
\end{dsldesc}

\begin{dslsig}
all(func: Member => Logical): Logical
\end{dslsig}
\begin{dsldesc}
    True if predicate \cc{func} holds for all selected members.
\end{dsldesc}

\begin{dslsig}
some(func: Member => Logical): Logical
\end{dslsig}
\begin{dsldesc}
    True if predicate \cc{func} holds for at least one selected member.
\end{dsldesc}

\begin{dslsig}
allEqual(func: Member => Any): Logical
\end{dslsig}
\begin{dsldesc}
    True if result of \cc{func} is the same for every selected member. In other
    words, the set of values yielded by \cc{func} for each selected member has at most
    one element.

    Typically used to ensure that components selected for a role all have the same value
    of some knowledge field, e.g., belong to the same team, have the same rank, etc.
\end{dsldesc}

\begin{dslsig}
allDifferent(func: Member => Any): Logical
\end{dslsig}
\begin{dsldesc}
    True if result of \cc{func} is different for every selected member. In other
    words, the set of values yielded by \cc{func} for each selected member is the
    same size as the set of selected members.

    Typically used to ensure that components selected for a role all differ in some
    knowledge field, e.g., no two components from the same team are picked.
\end{dsldesc}

\begin{dslsig}
disjointAfterMap[T,M](funcThis: Member => T, \\
~~~~~~~~~~~~~~~~~~~~~~other: MemberGroup[M], \\
~~~~~~~~~~~~~~~~~~~~~~funcOther: M => T): Logical
\end{dslsig}
\begin{dsldesc}
    True if, after converting the selected members of this and the other group to a
    common type \cc{T}, the resulting sets are disjoint.

    There are two types of usage for this function. One type is ensuring that two groups
    of same or similar types of components are partitioned according to some variable
    --- e.g., groups of workers disjoint over the projects they are working on.

    The other type is ensuring that two groups of different types of components do not
    mix with regard to some property. An example of this would be a sort of ``wolf,
    goat, cabbage'' scenario: the \cc{eaters} role in a \cc{Shore}
    ensemble must be disjoint with the \cc{foods} role, so that none of the
    eaters will eat any of the food.
\end{dsldesc}

%%%

\subsection{Collections of Member Groups}

Special behavior is defined for collections of \cc{MemberGroup}s, which are typically
obtained by mapping a collection of ensembles to one of their roles. The purpose of this
behavior is to support a common idiom: ensuring that membership of a role in multiple
ensembles does not overlap.

The following example assigns workers to teams in a way that no worker is assigned to
two teams:

\begin{lstlisting}[style=ensembles]
val workers: List[Worker] = /* ... */

class Team(val id: Int) extends Ensemble {
  val teamMembers = subsetOf(workers, _ > 0)
}

val teams = rules {
    for (i <- 1 to 4) yield new Team(i)
}

constraint { teams.map(_.teamMembers).allDisjoint }
constraint { teams.map(_.teamMembers).cardinality === workers.size }
\end{lstlisting}

\noindent
A collection of member groups has the following methods:

\begin{dslsig}
cardinality: Integer
\end{dslsig}
\begin{dsldesc}
    Cardinality of the collection, i.e., the total number of selected members across all
    groups in the collection.
\end{dsldesc}

\begin{dslsig}
allDisjoint: Logical
\end{dslsig}
\begin{dsldesc}
    True if all groups in the collection are disjoint, i.e., no member is selected in
    more than one group.
\end{dsldesc}

%%%

\subsection{\texttt{Policy} class}

The \cc{Policy} class represents the security policy in a single object, and provides
methods to initiate solving and examine its results.

The \cc{resolve()} method is the most straightforward way to interact with the policy
object:

\begin{dslsig}
resolve(): Boolean \\
resolve(limit: Long): Boolean
\end{dslsig}
\begin{dsldesc}
    Find a valid solution and record security actions. If no utility expression is
    defined, returns the first solution that satisfies all constraints. If an utility
    expression is defined, returns the maximum utility solution, or the best solution
    that could be found within a time limit. Returns \cc{true} if a solution was found,
    or \cc{false} if not.

    When \cc{limit} is specified, it is used as a time limit for the solving process, in
    milliseconds. If the method does not return before the time limit expires, the best
    solution found so far is recorded.

    This is an all-in-one method that performs all solving steps automatically. To
    customize the solving process, it is necessary to use the methods below.
\end{dsldesc}

\bigskip
\noindent
The following attributes are available as soon as a solution is attempted:

\begin{dslsig}
exists: Boolean
\end{dslsig}
\begin{dsldesc}
    True if a solution was found.
\end{dsldesc}

\begin{dslsig}
instance: Ensemble
\end{dslsig}
\begin{dsldesc}
    Reference to an instance of the root ensemble class. Through \cc{instance}, it is
    possible to examine role and sub-ensemble assignment.
\end{dsldesc}

\begin{dslsig}
actions: Iterable[Action]
\end{dslsig}
\begin{dsldesc}
    Generated list of security actions, collected from all sub-ensembles. Contains
    objects of type \cc{AllowAction}, \cc{DenyAction} and \cc{NotifyAction}.
\end{dsldesc}

\begin{dslsig}
solutionUtility: Int
\end{dslsig}
\begin{dsldesc}
    Total utility of the solution, if one exists. If the solution exists but has no
    utility expressions, this will return zero.
\end{dsldesc}

\begin{dslsig}
allows(actor: Component,\\
~~~~~~~action: String,\\
~~~~~~~subject: Component): Boolean
\end{dslsig}
\begin{dsldesc}
    Return true if \cc{actor} is permitted to \cc{action} on \cc{subject}.

    See section~\ref{dsl:c:access} for semantics of access control queries.
\end{dsldesc}

\bigskip
\noindent
In case a customization of the solving process is needed, the following methods are
available to run the solving step-by-step:

\begin{dslsig}
init() \\
init(limit: Long)
\end{dslsig}
\begin{dsldesc}
    Reset the solver, configure time limit, delete all solutions and recorded security
    actions, and prepare for finding a solution.

    Time limit is in milliseconds and applies across all subsequent solving runs. If
    the time limit expires while a call to \cc{solve()} is in progress, the solving
    process stops and \cc{solve()} returns \cc{false}. See section~\ref{dsl:c:time} for
    details.

    Must be called whenever the situation changes, otherwise the policy will reflect the
    previous state of ensembles, components, and the environment. In particular, must be
    called before the first call to \cc{solve()}.
\end{dsldesc}

\begin{dslsig}
solve(): Boolean
\end{dslsig}
\begin{dsldesc}
    Find one solution. Return \cc{true} if a valid solution is found, \cc{false}
    otherwise.
    
    This method can be called repeatedly to iterate over all solutions. If no solution
    is found, and a previously-found solution exists, it will still be accessible.
    
    If no utility expression is specified, repeated calls will yield successive
    solutions. With a utility expression, each successive call will find a solution with
    higher utility than the previous one. If no such solution exists, \cc{solve()} will
    return \cc{false}, even if other solutions exist with equal utility. To find the
    solution with maximum utility, the following idiom is used:\nopagebreak
\begin{lstlisting}[style=snippet]
while (policy.solve()) {}
\end{lstlisting}
\end{dsldesc}

\begin{dslsig}
commit()
\end{dslsig}
\begin{dsldesc}
    Commit current solution, generate security rules, and send notifications to
    components. Must be called before accessing \cc{actions} for a new solution.
\end{dsldesc}

