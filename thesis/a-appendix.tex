\chapter{Data Archive Contents}

The data archive accompanying this work is a copy of the \cc{tcoof-trust} GitHub
repository. Its up-to-date version can always be found at the following URL:
\href{https://github.com/matejcik/tcoof-trust}{\ttfamily https://github.com/matejcik/tcoof-trust}

The only additional content in the archive is the \cc{apidoc} directory, which contains
the generated Scaladoc API documentation. In a Git checkout, the API docs can be
generated with the command \dop{sbt doc}.

The following is a list of paths in the archive and descriptions of their contents.

\bigskip

\newenvironment{filepath}%
    {%
        \par\vspace{0.6em}\ttfamily\raggedright
    }%
    {%
        \vspace{-0.2em}
    }%

\begin{filepath}
README.md
\end{filepath}
\begin{dsldesc}
    Markdown-formatted \cc{README} for the Git repository. Please refer to this file for
    requirements to run the TCOOF-Trust framework and basic usage examples.
\end{dsldesc}

\begin{filepath}
build.sbt\\
project/
\end{filepath}
\begin{dsldesc}
    SBT (Scala Build Tool) build script and build properties. These files are required
    for building the project with \cc{sbt}, and for importing into IntelliJ IDEA, the
    preferred Scala IDE.
\end{dsldesc}

\begin{filepath}
apidoc/
\end{filepath}
\begin{dsldesc}
    Scaladoc API documentation in HTML format.

    The limitation of the generated API documentation is that it excludes non-public
    members. Unfortunately, that cannot be avoided with the Scaladoc tool. See the
    source code for documentation comments.
\end{dsldesc}

\begin{filepath}
python/all.sh
\end{filepath}
\begin{dsldesc}
    Shell script that installs the Python environment and regenerates graphs used in
    this work.
\end{dsldesc}

\begin{filepath}
python/Pipfile\\
python/Pipfile.lock
\end{filepath}
\begin{dsldesc}
    \cc{pipenv} configuration files. Contain list of Python package dependencies
    required for generating graphs.
\end{dsldesc}

\begin{filepath}
python/lunch.py\\
python/other.py\\
python/resultlib.py\\
python/variables.py
\end{filepath}
\begin{dsldesc}
    Individual Python scripts that generate the graphs. \cc{variables.py} generates
    graphs from section~\ref{eval:variables}, \cc{other.py} generates the simulation
    histogram and the timeout graph, \cc{lunch.py} generates all the rest.
    \cc{resultlib} is a small library of common functions.
\end{dsldesc}

\begin{filepath}
results/final/badsolver-growingprojects.log
\end{filepath}
\begin{dsldesc}
    Source data for figure~\ref{fig:badsolver}.
\end{dsldesc}

\pagebreak
\begin{filepath}
results/final/booleans.log\\
results/final/constraints.log\\
results/final/integers.log
\end{filepath}
\begin{dsldesc}
    Source data for figures \ref{fig:variables:time} and \ref{fig:variables:mem}
\end{dsldesc}

\begin{filepath}
results/final/moreprojects.log
\end{filepath}
\begin{dsldesc}
    Source data for figure~\ref{fig:workers-moreprojects}.
\end{dsldesc}

\begin{filepath}
results/final/morerooms-optimizing.log
\end{filepath}
\begin{dsldesc}
    Source data for figure~\ref{fig:workers-morerooms}.
\end{dsldesc}

\begin{filepath}
results/final/oneworker-params.log
\end{filepath}
\begin{dsldesc}
    Source data for figure~\ref{fig:workers-oneworker}.
\end{dsldesc}

\begin{filepath}
results/final/simulated.log
\end{filepath}
\begin{dsldesc}
    Source data for figure~\ref{fig:simulation}.
\end{dsldesc}

\begin{filepath}
results/final/timelimits.log
\end{filepath}
\begin{dsldesc}
    Source data for figure~\ref{fig:timelimits}.
\end{dsldesc}

\begin{filepath}
results/final/workercount-simple.log
\end{filepath}
\begin{dsldesc}
    Source data for figure~\ref{fig:workers-simple}.
\end{dsldesc}

\begin{filepath}
src/main/scala/cz/
\end{filepath}
\begin{dsldesc}
    Scala source files for the TCOOF-Trust framework, specifically the package
    \cc{cz.cuni.mff.d3s.trust}.
\end{dsldesc}

\begin{filepath}
src/main/scala/scenario/
\end{filepath}
\begin{dsldesc}
    Scala source files for the test measurement scenarios.
\end{dsldesc}

\begin{filepath}
src/test/scala/
\end{filepath}
\begin{dsldesc}
    Scala source files for the unit test suite.
\end{dsldesc}
