\chapter{Conclusion}
\label{conclusion}

In this work we have presented an ensemble-based approach to defining access control
policies. Security situations are specified in terms of ensembles and their roles, and
access control decisions are attached to the ensembles. This allows the policy to follow
the evolution of a dynamic system as its shape and composition changes over time.

We introduced a framework, TCOOF-Trust, which consists of two parts: a~policy
specification language, implemented as an internal DSL in Scala, and a runtime
environment for resolving the policies. The DSL can be used to specify ensembles, roles,
constraints on their membership, and attached security decisions. The runtime
environment processes the policy specification in the DSL, converts the problem to an
input for its internal CSP solver, and uses the results to determine ensemble
membership. Based on the solution and the rules attached to it, the framework can
respond to access control queries.

We have designed and specified detailed semantics for the behavior of the framework, and
extended the existing TCOOF-Trust prototype implementation to properly support them. A
clean API for embedding the framework in external projects is provided, and detailed
documentation was created for the CSP conversion process and for the implementation
details of the framework.

An example security scenario with dynamic access control requirements was used to
evaluate performance of the framework, both in synthetic and real-world-like
configurations. While the underlying problem is exponential by nature, results show that
the framework is capable of dealing with the complexity in a reasonable manner, and
scales well in practical deployments.
