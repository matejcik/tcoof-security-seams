\section{Core Concepts}
\label{dsl:concepts}

%%%

\subsection{Components}
\label{dsl:c:components}

Every entity that the runtime knows about needs to be represented as an object derived
from \cc{Component}. That means connected devices, locks, and even people, are treated
as components. They can be members of ensembles and inhabit roles.

A component represents the system's knowledge of an entity, but the framework has no
control over it. From its point of view, a component and its data are purely inputs.
This is also how we can represent people as components: as far as the runtime is
concerned, they are view-only.

Component instances don't need any methods, and in fact they should not have any. The
possible exception to this recommendation is ``computed knowledge''. Sometimes it is
useful to have a helper method that returns the result of some computation over the
component's attributes. For example, a worker component can have knowledge of its
location, and a method \cc{isInLunchRoom} that returns true if the location is a
lunchroom. The result of this method is still component knowledge, i.e., something we
know about the component, but does not need to be stored as a value.

Components should come from outside the ensemble definitions, preferably from outside
the scenario class. Component instances should never be created within an ensemble. All
knowledge about a component should be represented in its attributes.

%%%

\subsection{Ensembles}
\label{dsl:c:ensembles}

Every ensemble is represented as an instance of a subclass of \cc{Ensemble}. The
instance holds references to roles and their assigned components, sub-ensembles,
situation predicates and constraints.

If there is just one ensemble or sub-ensemble for a particular purpose, it can be
specified as a singleton with Scala's \cc{object} keyword. If more ensembles of the same
type are needed, a class is more appropriate. All instances of the class must be
explicitly created in the parent ensemble body. Every instance must also be registered,
using either \cc{rules} or \cc{ensembles} call:

\begin{lstlisting}[
    label=lst:rules,
    style=ensembles,
]
class Root extends Ensemble {
  object SubEnsembleObject extends Ensemble {
    // ...
  }
  class SubEnsemble(id: Int) extends Ensemble {
    // ...
  }

  val subEnsembles = for (i <- 1 to 5) yield new SubEnsemble(i)
  rules(subEnsembles)
  ensembles(SubEnsembleObject)
}
\end{lstlisting}

Section~\ref{dsl:c:situations} explains the difference between \cc{rules} and
\cc{ensembles}.

The class body can specify role membership, sub-ensembles, constraints, and situation
predicates. Each of these features is described in its own subsection.

%%%

\subsection{Ensemble Activation and Situations}
\label{dsl:c:situations}

An ensemble can be active or inactive. We also use the term \textit{selected} --- as in,
the ensemble is selected as a member of the parent ensemble. Ensembles registered with
the \cc{rules} function are active by default. When registered with \cc{ensembles}, the
framework dynamically determines whether the ensemble should be active or inactive,
based on constraints in its parents. 

An active ensemble takes part in the computation and all constraints specified in the
ensemble must be satisfied. Inactive ensembles are not considered, their roles have no
members, and all their sub-ensembles are also inactive --- i.e., an ensemble can only be
active if all its parents are active.

It is possible to specify a boolean \textit{situation predicate}. If the predicate
evaluates to false, the ensemble and all of its children are deactivated. Note that an
ensemble registered with \cc{ensembles} can still be inactive if its situation predicate
is true; it is a necessary condition, not a sufficient one.

The situation predicate is specified via the \cc{situation} construct. At most one
situation predicate can be used per ensemble; if more than one \cc{situation} is
defined, the last one is used and all others are ignored.

%%%

\subsection{Roles}
\label{dsl:c:roles}

Fundamentally, an ensemble is a collection of components assigned to distinct roles.

Some role assignments can be known in advance. Such roles don't need to be explicitly
created. It is a good practice to assign the component to a member variable of the
ensemble, but it is perfectly legal to use a variable from an outer scope directly.

Other roles are assigned dynamically by the framework. For these, it is necessary to
describe parameters of the assignment and a collection of candidates.

The function \cc{subsetOf} specifies that a particular role is a subset of a given
collection of components, or a subset of inhabitants of a different role. An optional
argument allows in-line specification of a constraint on the subset's cardinality, e.g.,
that no more than $N$ components must be selected.

As a shortcut, the \cc{oneOf} function defines that exactly one of the candidates will
be selected for the role. The \cc{allOf} function defines that all of the candidates
will be selected, effectively converting components directly to roles. This can be
useful in some cases, but it is usually not necessary to do it explicitly.

The function \cc{unionOf} can link several roles into a single group. This is useful in
situations where a constraint applies over many roles together and it would be difficult
to express it in terms of the individual roles.

Dynamically assigned roles are sets of components. A component can be a member of
multiple roles, but cannot be a member of the same role in the same ensemble more than
once.

The result of each of these functions is a \textit{role object}. After a solution is
found, this object can be used to access a list of the selected elements. At solving
time, however, the results are not yet available, and instead specialized methods of the
role object must be used. See section~\ref{dsl:r:membergroups} for a complete list of
available operations.

%%%

\subsection{Constraints}
\label{dsl:c:constraints}

A \textit{constraint} places arbitrary limits on the solution. Unlike situation
predicates, which are evaluated beforehand, constraints are applied during the search
for a solution. It is thus a good place to specify requirements that the solution must
fulfill. Constraints can be used to specify that membership of some role must be
disjoint with another role, mark one role's cardinality as strictly smaller than that of
another role, etc.

Constraint predicates are specified with a \cc{constraint} call. There can be multiple
\cc{constraint} calls in an ensemble, and all specified constraints must be fulfilled in
the solution. If a constraint in an active ensemble cannot be fulfilled, no solution
will be found.

Constraints in inactive ensembles are ignored.

Care must be taken when creating constraints over the results of \cc{subsetOf} and the
other role functions. Role objects define several methods, such as \cc{all}, \cc{sum},
and others, whose results are valid constraint objects. However, if the list of selected
members is accessed directly while specifying a constraint, an exception will be thrown.
See section~\ref{dsl:r:membergroups} for details.

%%%

\subsection{Solution Utility}
\label{dsl:c:utility}

It is possible to attach a utility expression to an ensemble. If present, the framework
will by default try to maximize the value of this expression when looking for solutions.
In other words, solutions with higher utility are preferred.

Utility expression can be specified with the \cc{utility} construct. Only one utility
expression can be specified per ensemble. If more than one \cc{utility} is used, only
the last one takes effect.

If multiple ensembles specify an utility expression, the total utility of the solution
is a simple sum of all the individual utilities. This is usually the desired behavior.
E.g., when calculating utilities for individual lunchrooms from the running example,
each room has its own utility, and we are interested in the sum of all rooms.

Sometimes this solution is insufficient, however. Maybe the total utility of an ensemble
is the average utility of each of its sub-ensembles. If that's the case, \cc{utility}
should not be used in the sub-ensembles, only in the parent. The sub-ensembles can
instead specify a method calculating the partial utility, which will then be used in
parent's utility expression:

\begin{lstlisting}[
    label=lst:subutility,
    style=ensembles,
]
class Root extends Ensemble {
  class SubEnsemble extends Ensemble {
    val someRole = subsetOf(members, _ < 4)
    def subUtility = someRole.cardinality * someRole.sum(_.weight)
  }

  val subEnsembles = for (_ <- 1 to 5) yield new SubEnsemble
  utility {
    subEnsembles.map(_.subUtility).reduce(_ + _) / subEnsembles.size
  }
}
\end{lstlisting}

Only active ensembles are counted towards the utility total.

Similar to role objects, utilities are not actual numbers. Methods like \cc{sum} are not
available and more elaborate calculations must be built from basic operators.

%%%

\subsection{Root Ensemble}
\label{dsl:c:root}

One ensemble must be designated as top-level, or root, in the scenario. This tells the
framework where to start with solving. The root ensemble is always active, and
\cc{situation} specified in the root has no effect.

It would be natural to specify the root ensemble as an \cc{object}, but due to
limitations of the DSL embedding, that is not possible. The root ensemble \textbf{must}
be a class, instantiated in a call to \cc{Policy.root}:

\begin{lstlisting}[
    label=lst:rootens,
    style=ensembles,
]
class Example {
  class Root extends Ensemble {
    // ...
  }
  val policy = Policy.root(new Root)
}
\end{lstlisting}

See section~\ref{impl:scala:byname} for explanation of this requirement.

Result of the \cc{Policy.root} call is a \cc{Policy} object. By convention, we always
store it in a member variable named \cc{policy} in a scenario class.

%%%

\subsection{Scenarios and Solving}
\label{dsl:c:solving}

Once a policy is specified, it is necessary to supply the actual \cc{Component}
instances and other contextual data. These are usually enclosed in a so-called
\textit{scenario class}.

In terms of code, the scenario class is a plain class and does not need to implement any
particular traits. Its purpose is code organization: it is a good practice to keep the
policy definition (i.e., the hierarchy of \cc{Ensemble}s) in the same scope with the
required context information. Having the context encapsulated within a class enables
use-cases such as multiple instances of the same policy for different buildings.

By convention, a \textit{policy instance} is stored in an attribute named \cc{policy}.

The policy instance, created with \cc{Policy.root} call, provides access to the
framework's solving functionality. The most basic feature is the \cc{resolve()} method,
which performs the computation, assigns members to roles, and generates access control
rules. The solution can be examined through the \cc{instance} member, which is a
reference to the instance of the root ensemble. The rules can be accessed through the
\cc{actions} member or queried via the \cc{allows()} method.

The framework does not monitor the situation continuously. The solution generated with
\cc{resolve()} is valid for the situation at the time it was called. If the situation
changes, component knowledge updates, etc., the user of the framework is responsible for
calling \cc{resolve()} again.

Every call to \cc{resolve()} runs from scratch. Previously established ensembles are
dissolved and members are assigned to roles without consideration for previous
assignments. This matches the operation of ensemble-based systems: ensembles are loose
coalitions that form and re-form based on current conditions. For handling of
persistence, refer to subsection~\ref{dsl:c:notifications}.

If no utility functions are specified, \cc{resolve()} will find and apply the first
solution that satisfies all the constraints. When utility functions are present, it will
find and apply a solution with the highest utility.

It is also possible to iterate over possible solutions manually. The \cc{init()} method
will reset solver state. After that, every call to \cc{solve()} will find a new
solution, viewable through \cc{instance}, or return \cc{false} if no more solutions are
found. Finally, a call to \cc{commit()} will apply access control rules from the current
solution.

The behavior is slightly different with utility functions; there is an added constraint
that each new solution must have higher utility than the one before it. That means that
it will not be possible to examine every valid solution, and the computation will stop
at the first solution that cannot be improved. Of course, finding out that a particular
solution cannot be improved might take a long time.

%%%

\subsection{Time Limits}
\label{dsl:c:time}

Some security policies can have an exponential number of solutions. Absent other
constraints, there are $2^N$ possible results of \cc{subsetOf} on $N$ members. Finding a
solution with maximum utility could therefore take a very long time.

To manage this issue, it is possible to configure a time limit after which the search
stops. As explained in the previous section, an optimizing solver iterates through valid
solutions, searching for those with increasing utility. By setting a time limit, we are
effectively declaring that we are interested not in the optimal solution, but the best
that can be found inside the limit.

Time limits are most useful in scenarios with utility expressions. They provide little
benefit when iterating over all valid solutions, as the search can usually be stopped
simply by not requesting another solution.

An optional parameter to \cc{init()} or \cc{resolve()} specifies a time limit in
milliseconds. The countdown starts at the time the method is called, and carries across
calls to \cc{solve()}. If the time limit expires while \cc{solve()} is running, it will
stop the computation and return false, and all subsequent calls will also return false
--- same as if no more solutions can be found.

%%%

\subsection{Access Control}
\label{dsl:c:access}

When the scenario is solved and component assignments are known, the runtime emits
specified access control rules. There are two available functions: \cc{allow} and
\cc{deny}. Both take three arguments: actor, action and subject. Actors and subjects can
be components, collections of components, or roles. Action must be a string.

Specifying a collection of actors or subjects is the same as specifying each actor and
each subject one by one. Specifying a role applies the rule to selected members of that
role.

If the ensemble is active, access control directives will be emitted. The framework
takes a default-deny approach. If a triplet does not exist in emitted directives, the
permission is denied. If an \cc{allow} triplet exists, the permission is granted, unless
a \cc{deny} triplet also exists. This way, it is possible to grant a~wide permission in
an ensemble, but refine it in a sub-ensemble or a different situation-specific ensemble.

%%%

\subsection{Notifications and Persistence}
\label{dsl:c:notifications}

Using the function \cc{notify}, it is possible to attach messages to components. This is
the only way the framework can affect components (which are usually autonomous entities
beyond our control). The notification feature serves two purposes.

First, it is possible to query the notifications inside an ensemble. This way it is
possible to persist earlier configurations. Consider this example:

\begin{lstlisting}[
    label=lst:notify,
    style=ensembles,
]
case class Reservation(room: Room) extends Notification

class SeatReservations(room: Room) extends Ensemble {
  val alreadyReserved = workers.filter(_.notified(Reservation(room)))
  val newlyReserved = oneOf(workers.filter(_.askingForReservation))

  allow(alreadyReserved, "enter", room)
  allow(newlyReserved, "enter", room)
  notify(newlyReserved, Reservation(room))
}
\end{lstlisting}

Workers that have reserved seats in previous runs will still be granted the \cc{"enter"}
permission on subsequent runs. If we didn't attach the notification, they would lose the
permission when the solution is rerun.

Second, a notification action is recorded as part of the generated access control rules.
Users of the system can listen for these notification actions and forward them to
components. This can be useful to, e.g., send a message to a worker's smartphone, to
inform them about their seat reservation.

Notification messages must implement trait \cc{Notification}. It is useful to define
them as case classes, so that it is possible to filter them by value, as demonstrated in
the example.

